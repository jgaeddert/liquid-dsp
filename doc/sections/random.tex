% 
% MODULE : random
%

\newpage
\section{random}
\label{module:random}
The random module in \liquid\ includes a comprehensive set of
random number generators useful for simulation of wireless
communications channels,
particularly for generating noise as well as fading channels.
This includes the
uniform,
normal,
circular (complex) Gaussian,
Rice-$K$, and
Weibull distributions.


% 
% Uniform
%
\subsection{Uniform}
\label{module:random:uniform}
%
\begin{equation}
\label{eqn:random:uniform:pdf}
    f_X(x) =
    \begin{cases}
        1 & \text{if $0 \leq x < 1$} \\
        0 & \text{else}.
    \end{cases}
\end{equation}

\subsection{Normal (Gaussian)}
\label{module:random:normal}
%
\begin{equation}
\label{eqn:random:normal:pdf}
    f_X(x;\sigma,\eta) =
        \frac{1}{\sigma \sqrt{2 \pi}}
        e^{-\left(x-\eta\right)^2/{2\sigma^2}}
\end{equation}
%
\liquid\ generates normal random variables using the Box-Muller method.
If $U_1$ and $U_2$ are uniform random variables with a distribution
defined by (\ref{eqn:random:uniform:pdf}), then
$X_1 = \sqrt{-2\ln(U_1)} \sin\left(2 \pi U_2\right)$ and
$X_2 = \sqrt{-2\ln(U_1)} \cos\left(2 \pi U_2\right)$
are independent normal random variables with a mean of zero and a unity
standard deviation ($X_1, X_2 \sim N(0,1)$).

% 
% Weibull
%
\subsection{Weibull}
\label{module:random:weibull}
The Weibull distrubtion has a probability denisty function defined by
%
\begin{equation}
\label{eqn:random:weibull:pdf}
    f_X(x;\alpha,\beta,\gamma) =
    \begin{cases}
        \frac{\alpha}{\beta}
        \left(
            \frac{x-\gamma}{\beta}
        \right)^{\alpha-1}
        \exp\Bigl\{
            -\left( \frac{x-\gamma}{\beta} \right)^\alpha
        \Bigr\}                         & \text{$x \ge \gamma$} \\
        0                               & \text{else}.
    \end{cases}
\end{equation}
%
where
$\alpha$ is the shape parameter,
$\beta$ is the scale parameter,
and
$\gamma$ is the threshold parameter.
%
\liquid\ generates Weibull random variables by inverting the cumulative
ditribution function, viz
%
\begin{equation}
\label{eqn:random:weibull:cdf}
    F_X(x;\alpha,\beta,\gamma) =
    \begin{cases}
        1 - \exp\left\{
            -\left(\frac{x-\gamma}{\beta}\right)^\alpha
        \right\} &                            x \ge \gamma \\
        0 &                                   \text{else}.
    \end{cases}
\end{equation}
%
Specifically if $U$ is uniform random variable with a distribution
defined by (\ref{eqn:random:uniform:pdf}) then
$X = \gamma + \beta\left[ \ln\left(1 - U\right) \right]^{1/\alpha}$
%
has a Weibull distribution defined by (\ref{eqn:random:weibull:cdf}).


% 
% Gamma
%
\subsection{Gamma}
\label{module:random:gamma}
%
\begin{equation}
\label{eqn:random:gamma:pdf}
    f_X(x;\alpha,\beta) =
    \begin{cases}
        \frac{
            x^{\alpha-1}
        }{
            \Gamma(\alpha) \beta^\alpha
        }
        e^{-x / \beta}                  & x \ge 0 \\
        0                               & \text{else}.
    \end{cases}
\end{equation}
%


% 
% Nakagami-m
%
\subsection{Nakagami-$m$}
\label{module:random:nakagamim}
The Nakagami-$m$ distribution is a versatile stochastic model for
modeling radio links \cite{Braun:1991} and has often been regarded as the
best distribution to model land mobile propagation due to its ability to
describe fading situations worse than Rayleigh, including one-sided
Gaussian \cite{Simon:1998}.
Empirical evidence regarding the efficacy the Nakagami-$m$ distribution
has on fading profiles been presented in \cite{Turin:1980, Suzuki:1977}.
Thus statistical inference of the Nakagami-$m$ fading parameters are of
interest in the design of adaptive radios such as optimized transmit
diversity modes \cite{Cavers:1999, Ko:2003} and adaptive modulation schemes
\cite{Catreux:2002}.
The Nakagami-$m$ probability density function is given by
\cite{Papoulis:2002}
%
\begin{equation}
\label{eqn:random:nakagamim:pdf}
    f_X(x;m,\Omega) =
    \begin{cases}
        \frac{2}{\Gamma(m)}
        \left( \frac{m}{\Omega} \right)^m
        x^{2m-1}
        e^{ -(m/\Omega)x^2}             & x \ge 0 \\
        0                               & \text{else}.
    \end{cases}
\end{equation}
%
where
$m \ge 1/2$ is the shape parameter and
$\Omega > 0$ is the spread parameter.


%
% Rice-K
%
\subsection{Rice-$K$}
\label{module:random:ricek}
%
\begin{equation}
\label{eqn:random:ricek:pdf}
    f_R(r;K,\Omega) = 
        \frac{2(K+1)r}{\Omega}
        \exp\left\{-K-\frac{(K+1)r^2}{\Omega}\right\}
        I_0\left( 2r\sqrt{\frac{K(K+1)}{\Omega}} \right)
\end{equation}
%
where $\Omega=E\left\{R^2\right\}$ is the average signal power and $K$
is the fading factor (shape parameter).
\liquid\ generates Rice-$K$ random variables using two independent
normal random variables.
%
\begin{eqnarray*}
    s       &=&     \sqrt{\frac{\Omega K}{K+1}}     \\
    \sigma  &=&     \sqrt{\frac{\Omega}{2(K+1)}}    \\
    X_0     &\sim&  N(0,\sigma)                     \\
    X_1     &\sim&  N(s,\sigma)                     \\
    R       &=&      \sqrt{X_0^2 + X_1^2}
\end{eqnarray*}


% 
% Data scramble
%
\subsection{Data scrambler}
\label{module:random:data_scrambler}
Physical layer synchronization of received waveforms relies on independent and
identically distributed underlying data symbols.
If the message sequence, however, is too repetitive
(such as '{\tt 00000....}' or '{\tt 11001100....}')
and the modulation scheme is BPSK, the synchronizer probably won't be able to
recover the symbol timing because adjacent symbols are too similar.
This is a result of spectral correlation introduced which can prevent physical
layer synchronization techniques from tracking or even acquisition.
Having said that, certain patterns {\em are} beneficial to synchronization and
actually help the receiver track to the incoming signal, however these are
usually only introduced as a preamble to a frame or packet where the receiver
knows what to expect.
It is therefore imperative to increase the short-term entropy of the
underlying data to prevent the receiver from losing its lock on the signal.
The data scrambler routine attempts to ``whiten'' the data sequence with a bit
mask in order to achieve maximum entropy.

\subsubsection{interface}
The data scrambler has two methods, described here:
\begin{description}
\item[{\tt scramble\_data()}]
    takes an input sequence of data and scrambles the bits by applying a
    periodic mask.
    The first argument is a pointer to the input data array; the second
    argument is its length (number of bytes).
\item[{\tt unscramble\_data()}]
    takes an input sequence of data and unscrambles the bits by applying the
    reverse mask applied by {\tt scramble\_data()}.
    Just like {\tt scramble\_data()}, the first argument is a pointer to the
    input data array; the second argument is its length (number of bytes).
\end{description}

See {\tt examples/scramble\_example.c} for a full example of the interface.

