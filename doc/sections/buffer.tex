% 
% MODULE : buffer
%
\section{buffer}
\label{module:buffer}

%
% gport
%
\subsection{gport}
\label{module:buffer:gport}
The {\tt gport} object implements a generic port to share data
between asynchronous threads.

There are two ways to access data: direct memory access and 
indirect (copied) memory acess.
Each has distinct advantages and distadvantages.
Depending upon your needs...
Regardless of which implementation you use, the model is equivalent:
a buffer of data (initially empty) is created.
The {\it producer} is the method in charge of writing to the buffer (or
``producing'' the data).
The {\it consumer} is the method in charge of reading the data from the buffer
(or ``consuming'' it).
The producer and consumer methods can exist on completely separate threads,
and do not need to be externally synchronized.
The {\it gport} object synchronizes the data between the ports.

\subsubsection{Direct Memory Access}
Using {\tt gport} via direct memory access is a multi-step process, equivalent
for both the producer and consumer threads.
For the sake of simplicity, we will describe the process for writing data to
the port on the producer side...

\begin{enumerate}
\item the producer requests a lock on the port
\item once the request is serviced, the port returns a pointer to the array of
      data...
\item the producer writes its data at this location
\item the producer then unlocks the port, allowing the consumer thread to read
      data from it.
\item this process is repeated as necessary
\end{enumerate}

\input{listings/gport.direct.example.c.tex}

\subsubsection{Indirect/Copied Memory Access}

\begin{itemize}
\item the producer 
\end{itemize}

\input{listings/gport.indirect.example.c.tex}

\subsubsection{Key differences between memory access modes}

\subsection{window}
used to implement a sliding window

{\tt window\_recreate()}
Similar to the standard C library's realloc(), recreate() extends an existing
window...
If the size of the new window is larger than the old one, the old values are
pushed to the end

8   : 1 2 3 4 5 6 7 8
16  : 0 0 0 0 0 0 0 0 1 2 3 4 5 6 7 8

8   : 1 2 3 4 5 6 7 8
4   : 5 6 7 8
