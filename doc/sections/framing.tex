% 
% MODULE : framing
%

\section{framing}
\label{module:framing}

\subsection{packetizer}
\label{module:framing:packetizer}

The liquid packetizer is a structure for abstracting multi-level forward
error-correction from the user.
The packetizer accepts a buffer of uncoded data bytes and adds a 32-bit
cyclic redundancy check (crc) before applying two levels of forward error-
correction and bit-level interleaving.  The user may choose any two 
supported FEC schemes (including none) and the packetizer object will
handle buffering and data management internally, providing a truly abstract
interface.  The same is true for the packet decoder which accepts an array
of possibly corrupt data and attempts to recover the original message using
the FEC schemes provided.  The packet decoder returns the validity of the
resulting CRC as well as its best effort of decoding the message.

The packetizer also allows for re-structuring if the user wishes to change
error-correction schemes or data lengths.  This is accomplished with the
{\tt packetier\_recreate()} method.

Here is a minimal example demonstrating the packetizer's most basic
functionality:
\input{listings/packetizer.example.c.tex}

See also: fec module, {\tt examples/packetizer\_example.c}


