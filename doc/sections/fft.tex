% 
% MODULE : fft (fast Fourier transform)
%

\newpage
\section{fft (fast Fourier transform)}
\label{module:fft}
%
The fft module in \liquid\ implements fast discrete Fourier transforms
including forward and reverse DFTs as well as real even/odd transforms.

\subsection{Complex Transforms}
\label{module:fft:dft}
Given a vector of complex time-domain samples
$\vec{x} = \left[x(0),x(1),\ldots,x(N-1)\right]^T$
the $N$-point forward discrete Fourier transform is computed as:
%
\begin{equation}
\label{eqn:fft:dft}
    X(k) = \sum_{i=0}^{N-1}{x(i) e^{-j 2 \pi k i/N}}
\end{equation}
%
Similarly, the inverse (reverse) discrete Fourier transform is:
\begin{equation}
\label{eqn:fft:idft}
    x(n) = \sum_{i=0}^{N-1}{X(i) e^{ j 2 \pi n i/N}}
\end{equation}
%
\liquid\ implements only the basic decimation-in-time FFT algorithm for
radix-2 transforms and the slow DFT method otherwise.
Internal methods requiring FFTs, however, will use the {\tt fftw3}
library \cite{fftw:web} if available.
The presence of {\tt fftw3.h} and {\tt libfftw3} are detected by the
configure script at build time.
If found, \liquid\ will link against {\tt fftw} for better performance
(it is, however, the fastest FFT in the west, you know).
If {\tt fftw} is unavailable, however, \liquid\ will use its own, slower
FFT methods for internal processing.
This eliminates {\tt libfftw} as an external dependency, but takes
advantage of it when available.

Internally, \liquid\ uses several algorithms for computing FFTs including
the standard decimation-in-time (DIT) for power-of-two transforms
    \cite[\S10-4]{Ziemer:1998},
the Cooley-Tukey mixed-radix method for composite transforms
    \cite{CooleyTukey:1965},
Rader's algorithm for prime-length transforms \cite{Rader:1968},
and the DFT given by (\ref{eqn:fft:dft}) for very small values of $N$.
The DFT requires $\ord\bigl(N^2\bigr)$ operations
and can be slow for even moderate sizes of $N$
which is why it is typically reserved for small transforms.
\liquid's strategy for computing FFTs is to recursively break the
transform into manageable pieces and perform the best method for each
step.
For example, a transform of length $N=128=2^7$ can be easily computed
using the standard DIT FFT algorithm which is computationally fast.
The Cooley-Tukey algorithm permits any factorable transform of size
$N=PQ$ to be computed with
$P$ transforms of size $Q$ and $Q$ transforms of size $P$.
For example, a transform of length $N=126$ can be computed using the
Cooley-Tukey algorithm with radices $P=9$ and $Q=14$.
%($N = 128 = 2 \cdot 3 \cdot 3 \cdot 7$)
Furthermore, each of these transforms can be further split using the
Cooley-Tukey algorithm (e.g. $9=3\cdot3$ and $14=2\cdot7$).
The smallest resulting transforms can finally be computed using the DFT
algorithm without much penalty.
For large transforms of prime length, \liquid\ uses Rader's algorithm
\cite{Rader:1968}
which permits any transform of prime length $N$ to be computed using
an FFT and an IFFT each of length $N-1$.
For example, Rader's algorithm can compute a 127-point transform using
the 126-point Cooley-Tukey transform (and its inverse) described above.%
\footnote{Rader actually gives an alternate algorithm by which any
          transform of prime length $N$ can be computed with an FFT and
          an IFFT of any length greater than $2N-4$.
          For example, the 127-point FFT could also be computed using
          computationally efficient 256-point DIT transforms.
          \liquid\ includes both algorithms and chooses the most
          appropriate one for the task.}
Through recursion, a tranform of any size can be decomposed into either
computationally efficient DIT FFTs, or combinations of small DFTs.
% 
% example:
%       [7440] = (2 * 2 * 2 * 2) * (3 * 5 * 31)
%         /\
%      16   465
% (DIT FFT)  /\
%          31  15
%     (Rader)  (C-T)
%        /       /\
%       30      3  5
%       /\   (DFT)(DFT)
%      6  5
%    (C-T)
%     /\
%    2  3
% (DFT)(DFT)
%
Consequently, \liquid\ can compute any transform in
$\ord\bigl(n\log(n)\bigr)$ operations.

An example of the interface for computing complex discrete Fourier
transforms is listed below.
Notice the stark similarity to {\tt libfftw3}'s interface.
%
\input{listings/fft.example.c.tex}
%
%\begin{figure}
%\centering
%\subfigure[FFT input (time series)] {
%  \includegraphics[trim = 0mm 0mm 0mm 0mm, clip, width=13cm]{figures.gen/fft_example_time}
%}
%\subfigure[FFT output (frequency response)] {
%  \includegraphics[trim = 0mm 0mm 0mm 0mm, clip, width=13cm]{figures.gen/fft_example_freq}
%}
%\caption{{\tt fft()} demonstration for a 201-point transform}
%\label{fig:module:fft}
%\end{figure}
%%
%An example of the FFT input relationship can be
%found in Figure~\ref{fig:module:fft}.

\subsection{Real even/odd DFTs}
\label{module:fft:r2r}
%
\liquid\ also implement real even/odd discrete Fourier transforms;
however these are not guaranteed to be efficient.
A list of the transforms and their descriptions is given below.
%
\subsubsection{{\tt FFT\_REDFT00} (DCT-I)}
\label{module:fft:r2r:REDFT00}
    \begin{equation}
    \label{eqn:fft:dct-I}
        X(k) = \frac{1}{2}\Bigl( x(0) + (-1)^k x(N-1) \Bigr) + 
               \sum_{n=1}^{N-2}{x(n) \cos\left(\frac{\pi}{N-1}nk\right) }
    \end{equation}

\subsubsection{{\tt FFT\_REDFT10} (DCT-II)}
\label{module:fft:r2r:REDFT10}
    \begin{equation}
    \label{eqn:fft:dct-II}
        X(k) =  \sum_{n=0}^{N-1}{
                    x(n) \cos\left[
                        \frac{\pi}{N}\left(n + 0.5\right)k
                    \right]
                }
    \end{equation}

\subsubsection{{\tt FFT\_REDFT01} (DCT-III)}
\label{module:fft:r2r:REDFT01}
    \begin{equation}
    \label{eqn:fft:dct-III}
        X(k) =  \frac{x(0)}{2} +
                \sum_{n=1}^{N-1}{
                    x(n) \cos\left[
                        \frac{\pi}{N}n\left(k + 0.5\right)
                    \right]
                }
    \end{equation}

\subsubsection{{\tt FFT\_REDFT11} (DCT-IV)}
\label{module:fft:r2r:REDFT11}
    \begin{equation}
    \label{eqn:fft:dct-IV}
        X(k) =  \sum_{n=0}^{N-1}{
                    x(n) \cos\left[
                        \frac{\pi}{N}
                        \left(n+0.5\right)
                        \left(k+0.5\right)
                    \right]
                }
    \end{equation}

\subsubsection{{\tt FFT\_RODFT00} (DST-I)}
\label{module:fft:r2r:RODFT00}
    \begin{equation}
    \label{eqn:fft:dst-I}
        X(k) =  \sum_{n=0}^{N-1}{
                    x(n) \sin\left[
                        \frac{\pi}{N+1}(n+1)(k+1)
                    \right]
                }
    \end{equation}

\subsubsection{{\tt FFT\_RODFT10} (DST-II)}
\label{module:fft:r2r:RODFT10}
    \begin{equation}
    \label{eqn:fft:dst-II}
        X(k) =  \sum_{n=0}^{N-1}{
                    x(n) \sin\left[
                        \frac{\pi}{N}(n+0.5)(k+1)
                    \right]
                }
    \end{equation}

\subsubsection{{\tt FFT\_RODFT01} (DST-III)}
\label{module:fft:r2r:RODFT01}
    \begin{equation}
    \label{eqn:fft:dst-III}
        X(k) =  \frac{(-1)^k}{2}x(N-1) + 
                \sum_{n=0}^{N-2}{
                    x(n) \sin\left[
                        \frac{\pi}{N}(n+1)(k+0.5)
                    \right]
                }
    \end{equation}

\subsubsection{{\tt FFT\_RODFT11} (DST-IV)}
\label{module:fft:r2r:RODFT11}
    \begin{equation}
    \label{eqn:fft:dst-IV}
        X(k) =  \sum_{n=0}^{N-1}{
                    x(n) \sin\left[
                        \frac{\pi}{N}(n+0.5)(k+0.5)
                    \right]
                }
    \end{equation}

An example of the interface for computing a discrete cosine transform
of type-III ({\tt FFT\_REDFT01}) is listed below.
%
\input{listings/fft_dct.example.c.tex}

%
%\subsection{{\tt mdct} (modified discrete cosine transform)}
%\label{module:fft:mdct}

%
% SPECTRAL PERIODOGRAM
%
\subsection{{\tt spgram} (spectral periodogram)}
\label{module:fft:spgram}

In harmonic analysis, the spectral periodogram is an estimate of the
spectral density of a signal over time.
For a signal $x(t)$, the spectral content at time $t_0$ may be estimated
over a time duration of $T$ seconds as
\[
    \hat{X}_{t_0}(\omega) =
        \frac{1}{T} \int_{0}^{T} { x(t-t_0)w(t)e^{-j\omega t} dt }
\]
where $w(t) = 0,\forall t \notin (0,T)$
is a temporal windowing function to smooth transitions
between transforms.
Typical windowing functions are the Hamming, Hann, and Kaiser windows
(see \S\ref{module:math:window} for a description and spectral
representation of available windowing functions in \liquid).
Internally, the {\tt spgram} object using the Hamming window.%
\footnote{Future development may permit the user to specify which
          windowing method is preferred.}
%
For a discretely-sampled signal $x(nT_s)$, the spectral content at time
index $p$ is
%    X(k) = \sum_{i=0}^{N-1}{x(i) e^{-j 2 \pi k i/N}}
\[
    \hat{X}_p(k) = 
        \frac{1}{N}
        \sum_{i=0}^{N-1}{
            x((i+p)T_s) w(iT_s) e^{-j 2 \pi k i/N}
        }
\]
%
which is simply the $N$-point discrete Fourier transform of the input
sequence indexed at $p$ with a shaping window applied.
%
\begin{figure}
\centering
  \includegraphics[trim = 0mm 0mm 0mm 0mm, clip, width=0.95\textwidth]{figures.pgf/fft_spgram_diagram}
\caption{Spectral periodogram functionality}
\label{fig:module:fft:spgram:diagram}
\end{figure}
%
Figure~\ref{fig:module:fft:spgram:diagram} depicts a spectral
periodogram for the discrete case
in which two overlapping transforms are taken with a delay between them.
The windowing function provides time localization at the expense of
frequency resolution.
Typically the length of the window is half the size of the transform,
and the delay is a quarter the size of the transform.

\liquid\ implements a discrete spectral periodogram with the
{\tt spgram} object.
%
Listed below is the full interface to the {\tt spgram} object.
%
\begin{description}
\item[{\tt spgram\_create(nfft,window\_len)}]
    creates and returns an {\tt spgram} object with a transform size of
    {\tt nfft} samples with a window of {\tt window\_len} samples.
    Internally,
    a Hamming window (see \S\ref{module:math:window:hamming})
    is used for spectral smoothing.
\item[{\tt spgram\_destroy(q)}]
    destroys an {\tt spgram} object, freeing all internally-allocated
    memory.
\item[{\tt spgram\_reset(q)}]
    clears the internal {\tt spgram} buffers.
\item[{\tt spgram\_push(q,*x,n)}]
    pushes $n$ samples of the array $\vec{x}$ into the internal buffer
    of an {\tt spgram} object.
\item[{\tt spgram\_execute(q,*X)}]
    computes the spectral periodogram output storing the result in the
    {\tt nfft}-point output array $\vec{X}$.
    The output array is of type {\tt float complex} and contain
    output of the FFT.
\end{description}
%
\begin{figure}
\centering
\subfigure[{\tt spgram} input (time series)] {
  \includegraphics[trim = 0mm 4mm 0mm 0mm, clip, width=13cm]{figures.gen/fft_spgram_time}
}
\subfigure[{\tt spgram} output (time-frequency response)] {
  \includegraphics[trim = 10mm 10mm 4mm 10mm, clip, width=13cm]{figures.gen/fft_spgram_freq}
}
\caption{Spectral periodogram {\tt spgram} demonstration for a frequency-modulated sinuosoid.}
\label{fig:module:fft:spgram}
\end{figure}
%
An example of the {\tt spgram} object can be
found in Figure~\ref{fig:module:fft:spgram} in which a
frequency-modulated sinusoid is generated and analyzed.
The frequency of the sinusoid changes over time and is clearly visible
in both plots.

