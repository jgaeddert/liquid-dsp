% 
% MODULE : fft (fast Fourier transform)
%

\section{{\tt fft} (fast Fourier transform)}
\label{module:fft}

Given a vector of complex time-domain samples
$\vec{x} = \left[x_0,x_1,\ldots,x_{N-1}\right]^T$\ldots

Forward transform:
\[
    X(k) = \sum_{i=0}^{N-1}{x(i) e^{-j 2 \pi k i/N}}
\]

Reverse transform:
\[
    x(n) = \sum_{i=0}^{N-1}{X(i) e^{ j 2 \pi n i/N}}
\]

\liquid\ implements only the basic decimation-in-time FFT algorithm for
radix-2 transforms and the slow DFT method otherwise.
Internal methods requiring FFTs, however, will use the {\tt fftw3} library
\cite{fftw:web} if available.
The presence of {\tt fftw3.h} and {\tt libfftw3} are detected by the configure
script at build time.
If found, \liquid\ will link against {\tt fftw} for better performance (it is,
however, the fastest FFT in the west, you know).
If {\tt fftw} is unavailable, however, \liquid\ will use its own, slower FFT
methods for internal processing.
This eliminates {\tt libfftw} as an external dependency, but takes advantage
of it when available.

\subsection{Real even/odd DFTs ({\it not yet implemented})}
\label{module:fft:r2r}

\begin{itemize}
\item {\tt FFT\_REDFT00} (DCT-I)
\item {\tt FFT\_REDFT10} (DCT-II)
\item {\tt FFT\_REDFT01} (DCT-III)
\item {\tt FFT\_REDFT11} (DCT-IV)

\item {\tt FFT\_RODFT00} (DST-I)
\item {\tt FFT\_RODFT10} (DST-II)
\item {\tt FFT\_RODFT01} (DST-III)
\item {\tt FFT\_RODFT11} (DST-IV)
\end{itemize}

\subsection{{\tt mdct} (modified discrete cosine transform)}
\label{module:fft:mdct}

\subsection{interface}
The basic usage to compute a one-dimensional FFT looks something like this:
%
\input{listings/fft.example.c.tex}
%
Notice the stark similarity to {\tt libfftw3}'s interface.

