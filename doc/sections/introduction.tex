%
% introduction
%

\section{Key points}
\begin{itemize}
\item open-source software-defined radio DSP algorithms
\item minimal dependence on external libraries
\item portable to embedded platforms
\item flexible configuration
\item targets cognitive radios and enabling technologies through
      flexible algorithmic development
\end{itemize}

\subsection{Features}
\begin{itemize}
\item automatic test scripts for validation and accuracy
\item benchmark tool for estimating computational speed on your machine
\end{itemize}


\section{Quick Start Guide}
To build:
\begin{verbatim}
cd /path/to/liquid/
./reconf
./configure
make
sudo make install
\end{verbatim}
You may also want to build and run the optional validation program via
\begin{verbatim}
make check
\end{verbatim}
and the benchmarking tool,
\begin{verbatim}
make bench
\end{verbatim}

\section{Tutorial}
Most of {\it liquid}'s signal processing elements are C structures which
retain the object's parameters, state, and other useful information.
The naming convention is
{\tt basename\_xxxt\_method} where
{\tt basename} is the base object name (e.g. {\tt interp}),
{\tt xxxt} is the type definition, and
{\tt method} is the object method.
The type definition describes respective output, internal, and input type.
Types are usually {\tt f} to denote standard 32-bit {\it floating point}
precision values and can either be represented as {\tt r} (real) or {\tt c}
(complex).
For example, a {\tt dotprod} (vector dot product) object with complex input
and output types but real internal coefficients operating on 32-bit
floating-point precision values is {\tt dotprod\_crcf}.

Most objects have at least four standard methods:
{\tt create()},
{\tt destroy()},
{\tt print()},
and
{\tt execute()}.
Certain object also implement a {\tt recreate()} method which operates similar
to that of {\tt realloc()} in C.
A few points to note:
\begin{enumerate}
\item While the objects do retain internal memory, they operate on external
arrays defined by the user. That is... It is strictly up to the user to manage
his/her own memory.
\item ...
\end{enumerate}

\section{Learning by example}
The {\tt examples/} subdirectory includes numerous examples for nearly all the
signal processing components.

\section{Sandbox}

\subsection{Why C?}
A commonly asked question is ``why C and not C++?''
The answer is simple: {\em portability}.
Our aim is to provide a lightweight DSP library for software-defined radio
that does not rely on a myriad of dependencies.
While C++ is a fine language for many purposes (and theoretically runs just as
fast as C), it is not as portable to embedded platforms as C.
Furthermore, the majority of functions simply perform complex operations on a
data sequence and do not require a high-level object-oriented programming
interface.
This we will leave to framework developers and interface builders.

While a number of signal processing elements in \liquid\ use structures, these
are simply to save the internal state of the object.
For instance, a {\tt firfilt\_crcf} (finite impulse response filter) object
is just a structure which contains--among other things--the filter taps
(coefficients) and an input buffer.
For the most part, C++ polymorphic data types and abstract base classes are
unnecessary for basic signal processing, and primarily serve to reduce the
code base of a project.
Furthermore, while templates can certainly be useful for library development,
their benefits are of limited use to signal processing and can be circumvented
through the use of pre-processor macros at the gain of targeting more embedded
processors.

The C programming language has a rich history in system programming--
specifically targeting embedded applications--and is the basis behind many
well-known projects including the linux kernel and the python programming
language.

\subsection{Data Types}
The majority of signal processing for SDR is performed at complex baseband.
Complex numbers are handled in \liquid\ by defining data type
{\tt liquid\_float\_complex} which is binary compatible with the standard
C math type {\tt float complex} and C++ type {\tt std::complex<float>}.

Fixed-point data types are defined in the \liquidfpm\ library (see XXX).

\subsection{Building/Linking with C++}
Although \liquid\ is written in C, it can be seamlessly compiled and linked
with C++ source files.
Here is an example:
\input{listings/nco.c++.tex}

\section{History}
\liquid\ was created by J. Gaeddert out of necessity to perform complex
digital signal processing algorithms on embedded devices
without relying on dealing
with proprietary and otherwise cumbersome frameworks.
This was a critical step in his PhD thesis to adapt DSP algorithms in
cognitive dynamic-spectrum radios to optimally manage finite radio resources.
Was he successful?
Put it this way: at the time this document was written he still has not
graduated.\footnote{Come back in a year and ask again... {\em sigh}}

