% 
% MODULE : audio
%

\newpage
\section{audio}
\label{module:audio}
The audio module in \liquid\ provides several objects and functions for
compressing, digitizing, and manipulating audio signals.
This is particularly useful for encoding audio data for wireless
communications.

\subsection{{\tt cvsd} (continuously variable slope delta)}
\label{module:audio:cvsd}
Continuously variable slope delta (CVSD) source encoding is used for data
compression of audio signals.
CVSD is a lossy compression whose quality is directly related to the sampling
frequency and is generally most practical for speech applications.
It is a form of delta modulation where $\Delta$ (the step size) is changed
continuously to minimize slope-overload distortion \cite[p. 131]{Proakis:2001}.
The output bit stream has a rate equal to that of the sampling frequency.
It is considered to be a moderate compromise between quality and complexity.

\subsubsection{Theory}
\label{module:audio:cvsd:theory}
The algorithm attempts to dynamically adjust the value of $\Delta$
to track to the input signal.
As with regular delta modulation algorithms,
if the decoded reference signal exceeds the input (the error signal is
negative), a binary {\tt 0} is sent and $\Delta$ is subtracted from the
reference, otherwise a binary {\tt 1} is sent and $\Delta$ is added.
However CVSD observes the previous $N$ transmitted bits are stored in a
buffer $\hat{\vec{b}}$;
$\Delta$ is increased by $\zeta$ if they are equal and decreased
otherwise.
This improves the dynamic range of the encoder over fixed-delta
modulation encoders.
%
A summary of the encoding procedure can be found in
Algorithm~\ref{alg:module:audio:cvsd:encoder}.
%
% ALGORITHM : CVSD encoder
\begin{algorithm}[H]
\caption{CVSD encoder algorithm}
\label{alg:module:audio:cvsd:encoder}
\algsetup{linenosize=\footnotesize}
\algsetup{linenodelimiter=:}
\algsetup{indent=2em}
\begin{algorithmic}[1]
\STATE $\vec{x} \leftarrow \{x_0,x_1,x_2,\ldots\}$  \COMMENT{input audio samples}
\STATE $v_0 \leftarrow 0$                           \COMMENT{initial output reference}
\STATE $\Delta_0 \leftarrow \Delta_{min}$           \COMMENT{initialize step size}
\STATE $\hat{\vec{b}}_0 \leftarrow \{0,0,\ldots,0\}$\COMMENT{initialize $N$-bit buffer}

\FOR{$k=0,\,1,\,2,\,\ldots$}
    \STATE $b_k \leftarrow \begin{cases} 0 & v_k  > x_k \\ 1 & \text{else}\end{cases}$
        \COMMENT{compute output bit}
    \STATE $\hat{\vec{b}}_k \leftarrow  \{\hat{b}_1,\hat{b}_2,\ldots,\hat{b}_{N-1},b_k\}$
        \COMMENT{append output bit to end of buffer}
    \STATE $m \leftarrow \sum_{i=0}^{N-1}{\hat{\vec{b}}_i}$
        \COMMENT{compute sum of last $N$ bits}
    \STATE $\Delta_k \leftarrow \begin{cases}\Delta_{k-1}\zeta & m = 0, m=N \\ \Delta_{k-1}/\zeta & \text{else}\end{cases}$
        \COMMENT{adjust step size}
    \STATE $v_{k+1} \leftarrow v_k + (-1)^{1-b_k} \Delta_k$
        \COMMENT{adjust reference value}
\ENDFOR
\end{algorithmic}
\end{algorithm}
%

The decoder reverses this process;
by retaining the past $N$ bit inputs in a buffer $\hat{\vec{b}}$,
the value of $\Delta$ can be adjusted appropriately.
A summary of the decoding procedure can be found in
Algorithm~\ref{alg:module:audio:cvsd:decoder}.
%
% ALGORITHM : CVSD encoder
\begin{algorithm}[H]
\caption{CVSD decoder algorithm}
\label{alg:module:audio:cvsd:decoder}
\algsetup{linenosize=\footnotesize}
\algsetup{linenodelimiter=:}
\algsetup{indent=2em}
\begin{algorithmic}[1]
\STATE $\vec{b} \leftarrow \{b_0,b_1,b_2,\ldots\}$  \COMMENT{input bit samples}
\STATE $v_0 \leftarrow 0$                           \COMMENT{initial output reference}
\STATE $\Delta_0 \leftarrow \Delta_{min}$           \COMMENT{initialize step size}
\STATE $\hat{\vec{b}}_0 \leftarrow \{0,0,\ldots,0\}$\COMMENT{initialize $N$-bit buffer}

\FOR{$k=0,\,1,\,2,\,\ldots$}
    \STATE $\hat{\vec{b}}_k \leftarrow  \{\hat{b}_1,\hat{b}_2,\ldots,\hat{b}_{N-1},b_k\}$
        \COMMENT{append output bit to end of buffer}
    \STATE $m \leftarrow \sum_{i=0}^{N-1}{\hat{\vec{b}}_i}$
        \COMMENT{compute sum of last $N$ bits}
    \STATE $\Delta_k \leftarrow \begin{cases}\Delta_{k-1}\zeta & m = 0, m=N \\ \Delta_{k-1}/\zeta & \text{else}\end{cases}$
        \COMMENT{adjust step size}
    \STATE $v_{k+1} \leftarrow v_k + (-1)^{1-b_k} \Delta_k$
        \COMMENT{adjust reference value}
    \STATE $y_k \leftarrow v_k$
        \COMMENT{set output value}
\ENDFOR
\end{algorithmic}
\end{algorithm}
%

%
\subsubsection{Pre-/Post-Filtering}
\label{module:audio:cvsd:filtering}
To preserve the signal's integrity the encoder applies a pre-filter
to emphasize the high-frequency information of the signal before the
encoding process.
The pre-filter is a simple 2-tap FIR filter defined as
%
\begin{equation}
    H_{pre}(z) = 1 - \alpha z^{-1}
\end{equation}
%
where $\alpha$ controls the amount of emphasis applied.
Typical values fore pre-emphasis are $0.92 < \alpha < 0.98$;
setting $\alpha=0$ completely disables this emphasis.
%
This process is reversed on the decoder by applying the inverse of
$H_{pre}(z)$ as a low-pass de-emphasis filter:
%
\begin{equation}
    H_{pre}^{-1}(z) = 
        \frac{ 1 }{ 1 - \alpha z^{-1} }
\end{equation}
%
Additionally, the decoder adds a DC-blocking filter to reject any
residual offset caused by the decoding process.
By itself the DC-blocking filter has a transfer function
%
\begin{equation}
    H_{0}(z) = 
        \frac{ 1 - z^{-1} }{ 1 - \beta z^{-1} }
\end{equation}
%
where $\beta$ controls the cut-off frequency of the filter and is
typically set very close to 1.
The default value for $\beta$ in \liquid\ is 0.99.
The full post-emphasis filter is therefore
%
\begin{equation}
    H_{post}(z) = 
    H_{pre}^{-1}(z) H_0(z) =
        \frac{
            1 - z^{-1}
        }{
            1 - (\alpha + \beta) z^{-1} + \alpha\beta z^{-2}
        }
\end{equation}
%

\subsubsection{Interface}
\label{module:audio:cvsd:interface}


The {\tt cvsd} object in \liquid\ allows the user to select both $\zeta$
as well as $N$, the number of repeated bits observed before $\Delta$ is
updated.
The combination of these values with the sampling rate yields a speech
compression algorithm with moderate quality.
Listed below is the full interface to the {\tt cvsd} object:
%
\begin{description}
\item[{\tt cvsd\_create(N,zeta,alpha)}]
    creates an {\tt agc} object with parameters $N$, $\zeta$, and
    $\alpha$.
\item[{\tt cvsd\_destroy(q)}]
    destroys a {\tt cvsd} object, freeing all internally-allocated
    memory and objects.
\item[{\tt cvsd\_print(q)}]
    prints the {\tt cvsd} object's internal parameters to the standard
    output.
\item[{\tt cvsd\_encode(q,sample)}]
    encodes a single audio sample, returning the encoded bit.
\item[{\tt cvsd\_decode(q,bit)}]
    decodes and returns a single audio sample from an input bit.
\item[{\tt cvsd\_encode8(q,samples,byte)}]
    encodes a block of 8 samples returning the result in a single byte.
\item[{\tt cvsd\_decode8(q,byte,samples)}]
    decodes a block of 8 samples from an encoded byte.
\end{description}

\subsubsection{Example}
\label{module:audio:cvsd:example}

Here is a basic example of the {\tt cvsd} object in \liquid:
%
\input{listings/cvsd.example.c.tex}
%
A demonstration of the algorithm can be seen in
Figure~\ref{fig:module:audio:cvsd} where the encoder attempts to track to an
input sinusoid.
Notice that the encoder sometimes overshoots the reference signal.
This distortion results in degradations, particularly in the upper frequency
bands.
%
\begin{figure}
\centering
  \includegraphics[trim = 0mm 0mm 0mm 0mm, clip, width=13cm]{figures.gen/audio_cvsd}
\caption{
    {\tt cvsd} example encoding a windowed sum of sine functions
    with $\zeta=1.5$, $N=2$, and $\alpha=0.95$.}
\label{fig:module:audio:cvsd}
\end{figure}
%
A more detailed example is given in
{\tt examples/cvsd\_example.c}
under the main \liquid\ project directory.
