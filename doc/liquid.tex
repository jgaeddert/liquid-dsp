% ------------------------------------------------------------------
%
%   TITLE: LIQUID documentation
% AUTHORS: Joseph D. Gaeddert et al.
% CREATED: February 7, 2009
% REVISED: 
%     URL: http://computing.ece.vt.edu/~jgaeddert/
%          https://ganymede.ece.vt.edu/trac/liquid/
%
% ------------------------------------------------------------------
 
\documentclass[11pt,twoside]{article}

% ------------------- DOCUMENT VARIABLES -------------------
\setlength{\textwidth}{6.5in}
\setlength{\textheight}{8.5in}
\setlength{\evensidemargin}{0in}
\setlength{\oddsidemargin}{0in}
\setlength{\topmargin}{0in}
%\setlength{\parindent}{0pt}
%\setlength{\parskip}{0.1in}


% ------------------- BIBLIOGRAPHY STYLE -------------------
\usepackage{natbib}
\bibliographystyle{plain} % note the change here
\bibpunct{[}{]}{,}{n}{}{;}  % define bibliography punctuation
% The six mandatory arguments for \bibpunct are:
%   1. opening bracket: '(', '[', '{', or '<'
%   2. closing bracket: ')', ']', '}', or '>'
%   3. separator between multiple citations: ';' or ','
%   4. citation style: 'n' for numerical style, 's' for numerical superscript
%      style, or 'a' for author year style
%   5. punctuation between the author names and the year
%   6. punctuation between years or numbers when common author lists are suppressed: ',' or ';' 

% ------------------- GRAPHICS PACKAGES -------------------
%\usepackage{epsf}
%\usepackage{graphicx}
\ifx\pdfoutput\undefined
\usepackage{graphicx}
\else
\usepackage[pdftex]{graphicx}
\fi
\usepackage{epsfig}
\usepackage{epstopdf}
\usepackage{colortbl}
\usepackage{color}

%\usepackage{tabularx}
\usepackage{ctable} % \toprule, \midrule, \bottomrule
\setlength{\heavyrulewidth}{0.1em} % modify thickness of \toprule, \bottomrule
\newcommand{\otoprule}{\midrule[\heavyrulewidth]}
\usepackage{subfigure}
\usepackage{multirow} % for tables
%\usepackage{fancyhdr}
\usepackage{amsmath}
\usepackage{amssymb}
\usepackage{listings}
\usepackage{fancyvrb}
\usepackage{acronym}

% defines compact itemize*, enumerate*, and description* environments
% http://www.ctan.org/tex-archive/help/Catalogue/entries/mdwtools.html
%\usepackage{mdwlist}

% cleveref package, http://www.ctan.org/tex-archive/macros/latex/contrib/cleveref/
%\usepackage[capitalize]{cleveref}

% tabular: \hline (thin) \hline\hline (thick)
% from: http://www.faqs.org/faqs/tex-faq/ (see #44)
\setlength{\doublerulesep}{\arrayrulewidth}

%--------- NEW COMMANDS -------------------
\newcommand{\NA}{\scriptsize{NA}}
\newcommand{\E}{\scriptsize{E}}
\newcommand{\tRMS}{\ensuremath{\tau_{rms}}}
\newcommand{\tmED}{\ensuremath{\bar{\tau}}}
\newcommand{\Np}{\ensuremath{N_p}}
\newcommand{\ua}{\ensuremath{\uparrow}}
\newcommand{\da}{\ensuremath{\downarrow}}
\newcommand{\sinc}{\textup{sinc}}
\newcommand{\etal}{{\it et al.}}
\newcommand{\wavt}{W@VT}
\renewcommand{\vec}[1]{\boldsymbol{#1}}
\newcommand{\ciren}{C{\sc iren}}
\newcommand{\ord}{\mathcal{O}}
\newcommand{\liquid}{{\it liquid}}
\newcommand{\liquidfpm}{{\it liquid-fpm}}

%--------- CAPTION OPTIONS -------------------
\usepackage[small,bf]{caption}
\setcaptionwidth{15cm}
\setlength{\belowcaptionskip}{0.5cm}

%%%%%%%%%%%%%%%%%%%%%%%%%%%%%%%%%%%%%%%%%%%%%%%%%%%%%%%%%%%%%%%%%%%%%
%
%             MAIN DOCUMENT
%
%%%%%%%%%%%%%%%%%%%%%%%%%%%%%%%%%%%%%%%%%%%%%%%%%%%%%%%%%%%%%%%%%%%%%

\begin{document}

% ------------------- DEFINE LISTINGS -------------------
\input{highlight.sty}


%%%%%%%%%%%%%%%%%%%%%%%%%%%%%%%%%%%%%%%%%%%%%%%%%%%%%%%%%%%%%%%%%%%%%
%
%             TITLE PAGE
%
%%%%%%%%%%%%%%%%%%%%%%%%%%%%%%%%%%%%%%%%%%%%%%%%%%%%%%%%%%%%%%%%%%%%%
\thispagestyle{empty}
\pagenumbering{roman}
\begin{center}

{\huge\it liquid} \\
Software-defined radio digital signal processing library

\vfill

User's Manual for Version 1.0.0

\vfill

Joseph D. Gaeddert

\vfill

\today \\
Blacksburg, Virginia

\vfill

{\it Keywords:}
polyphase filterbanks,
OFDM/OQAM,
power consumption,
cognitive radio,
software-defined radio,
dynamic spectrum access
\\

\end{center}

\pagebreak
%%%%%%%%%%%%%%%%%%%%%%%%%%%%%%%%%%%%%%%%%%%%%%%%%%%%%%%%%%%%%%%%%%%%%
%
%             TABLE OF CONTENTS
%
%%%%%%%%%%%%%%%%%%%%%%%%%%%%%%%%%%%%%%%%%%%%%%%%%%%%%%%%%%%%%%%%%%%%%
\tableofcontents
\pagebreak

%\listoffigures
%\pagebreak

%\listoftables
%\pagebreak

%\section*{List of Abbreviations}
%\begin{acronym}
%  \acro{AWGN}{additive white Gauss noise}
%\end{acronym}

\pagenumbering{arabic}

%%%%%%%%%%%%%%%%%%%%%%%%%%%%%%%%%%%%%%%%%%%%%%%%%%%%%%%%%%%%%%%%%%%%%
%
%             Introduction
%
%%%%%%%%%%%%%%%%%%%%%%%%%%%%%%%%%%%%%%%%%%%%%%%%%%%%%%%%%%%%%%%%%%%%%

\newpage
\part{Introduction to \liquid }
\label{part:intro}

\bigskip
\noindent
To get you started with using \liquid\ signal processing library...

% TODO throw up some fancy graphic here


%
% introduction
%

\newpage
\section{Background and History}

\liquid\ is a free and open-source digital signal processing (DSP) library
designed specifically for software-defined radios on embedded platforms.
The aim is to provide a lightweight DSP library that does not rely on a myriad
of external dependencies or proprietary and otherwise cumbersome frameworks.
%
All signal processing elements are designed to be flexible, scalable, and
dynamic, including filters, filter design, oscillators, modems, synchronizers,
and complex mathematical operations.
The source for \liquid\ is written entirely in C so that it can be
compiled quickly with a low memory footprint.

\liquid\ was created by J. Gaeddert out of necessity to perform complex
digital signal processing algorithms on embedded devices
without relying on dealing
with proprietary and otherwise cumbersome frameworks.
This was a critical step in his PhD thesis to adapt DSP algorithms in
cognitive dynamic-spectrum radios to optimally manage finite radio resources.


% Key points
%   * open-source software-defined radio DSP algorithms
%   * minimal dependence on external libraries
%   * portable to embedded platforms
%   * flexible configuration
%   * targets cognitive radios and enabling technologies through
%     flexible algorithmic development
%
% Features
%   * automatic test scripts for validation and accuracy
%   * benchmark tool for estimating computational speed on your machine
%

\section{Quick Start Guide}
\label{section:quickstart}
The library can be easily built from source and is available from
several places.
The two most typical means of distribution are a compressed archive
(a {\em tarball}) and cloning the source repository.
Tarballs are generated with each stable release and are recommended for
users not expecting bleeding edge development.
Users wanting the very latest version (in addition to every other
version) should clone the \liquid\ {\tt git} repository.

\subsection{Building from an archive}
\label{section:quickstart:build_from_tarball}
\label{xxx}
Download the compressed archive {\tt liquid-dsp-v.v.v.tar.gz} to your
local machine where {\tt v.v.v} denotes the version of the release.
% Check the validity of the tarball using MD5...
%
% generate:
%   $ md5sum liquid-dsp-v.v.v.tar.gz > liquid-dsp.md5
%
% check:
%   $ md5sum --check liquid-dsp.md5
%
% You should see a message verifying the file:
%   liquid-dsp-v.v.v.tar.gz: OK
%
% If it fails, do no unpack.
Unpack the tarball
%
\begin{verbatim}
    $ tar -xvf liquid-dsp-v.v.v.tar.gz
\end{verbatim}
%
Move into the directory and run the configure script and make the
library.
%
\begin{verbatim}
    $ cd liquid-dsp-v.v.v
    $ ./configure
    $ make
    # make install
\end{verbatim}

\subsection{Building from the {\tt git} repository}
\label{section:quickstart:build_from_git}
\liquid\ uses {\tt git} \cite{git:web} for version control, a free and
open-source.
The benefits of {\tt git} over many other version control systems are
numerous and the list is too long to give here;
however one of the most important aspects is that each clone holds a
copy of the entire repository with a complete history and record of each
revision.
The main repository for \liquid\ is hosted online by {\em github}
\cite{github:web} and can be cloned on your local machine via
%
\begin{verbatim}
    $ git clone git://github.com/jgaeddert/liquid-dsp.git
\end{verbatim}
%
Move into the directory and build as before with the archive,
but with the additional bootstrapping step:
%
\begin{verbatim}
    $ cd liquid-dsp.git
    $ ./reconf
    $ ./configure
    $ make
    # make install
\end{verbatim}

\subsection{Additional targets}
\label{section:quickstart:additional_targets}
%
You may also want to build and run the optional validation program
(see Section~\ref{section:installation:targets:autotests}) via
\begin{verbatim}
    $ make check
\end{verbatim}
and the benchmarking tool
(see Section~\ref{section:installation:targets:benchmarks})
\begin{verbatim}
    $ make bench
\end{verbatim}
%
A comprehensive list of signal processing examples are given in the
{\tt examples} directory and can be built as
\begin{verbatim}
    $ make examples
\end{verbatim}
%
Each example can be built individually by directly targeting its binary
(e.g. {\tt make examples/cvsd\_example}).

\section{Tutorial}
\label{section:tutorial}
Most of \liquid's signal processing elements are C structures which
retain the object's parameters, state, and other useful information.
The naming convention is
{\tt basename\_xxxt\_method} where
{\tt basename} is the base object name (e.g. {\tt interp}),
{\tt xxxt} is the type definition, and
{\tt method} is the object method.
The type definition describes respective output, internal, and input type.
Types are usually {\tt f} to denote standard 32-bit {\it floating point}
precision values and can either be represented as {\tt r} (real) or {\tt c}
(complex).
For example, a {\tt dotprod} (vector dot product) object with complex input
and output types but real internal coefficients operating on 32-bit
floating-point precision values is {\tt dotprod\_crcf}.

Most objects have at least four standard methods:
{\tt create()},
{\tt destroy()},
{\tt print()},
and
{\tt execute()}.
Certain objects also implement a {\tt recreate()} method which operates
similar to that of {\tt realloc()} in C and are used to restructure or
reconfigure an object without completely destroying it and creating it again.
Typically, the user will create the signal processing object independent of
the external (user-defined) data array.
The object will manage its own memory until its {\tt destroy()} method is
invoked.
A few points to note:
\begin{enumerate}
\item The object is only used to maintain the state of the signal processing
      algorithm.
      For example, a finite impulse response filter
      (Section~\ref{module:filter:firfilt}) needs to retain the filter
      coefficients and a buffer of input samples.
      Certain algorithms which do not retain information (those which are
      memoryless) do not use objects.
      For example, {\tt design\_rrc\_filter()}
      (Section~\ref{module:filter:firdes:rrcos})
      calculates the coefficients of a root raised-cosine filter, a processing
      algorithm which does not need to maintain a state after its completion.
\item While the objects do retain internal memory, they typically operate on
      external user-defined arrays.
      As such, it is strictly up to the user to manage his/her own memory.
      Shared pointers are a great way to cause memory leaks, double-free bugs,
      and severe headaches.
%\item ...
\end{enumerate}

\section{Learning by example}
While this document contains numerous examples listed in the text, they are
typically condensed to demonstrate only the interface.
The {\tt examples/} subdirectory includes more extensive demonstrations and
numerous examples for all the signal processing components.
Many of these examples write an output file which can be read by
octave~\cite{octave:web} to display the results.
For a brief description of each of these examples, see {\tt examples/README}.

\subsection{Why C?}
A commonly asked question is ``why C and not C++?''
The answer is simple: {\em portability}.
Our aim is to provide a lightweight DSP library for software-defined radio
that does not rely on a myriad of dependencies.
While C++ is a fine language for many purposes (and theoretically runs just as
fast as C), it is not as portable to embedded platforms as C.
Furthermore, the majority of functions simply perform complex operations on a
data sequence and do not require a high-level object-oriented programming
interface.
The importance in object-oriented programming is the techniques used, not the
languages describing it.

While a number of signal processing elements in \liquid\ use structures, these
are simply to save the internal state of the object.
For instance, a {\tt firfilt\_crcf} (finite impulse response filter) object
is just a structure which contains--among other things--the filter taps
(coefficients) and an input buffer.
This simplifies the interface to the user; one only needs to ``push'' elements
into the filter's internal buffer and ``execute'' the dot product when
desired.
This could also be accomplished with classes, a construct specific to C++ and
other high-level object-oriented programming languages, however,
for the most part, C++ polymorphic data types and abstract base classes are
unnecessary for basic signal processing, and primarily just serve to reduce
the code base of a project.
Furthermore, while C++ templates can certainly be useful for library development,
their benefits are of limited use to signal processing and can be circumvented
through the use of pre-processor macros at the gain of targeting more embedded
processors.
Under the hood, the C++ compiler's pre-processor expands templates and classes
before actually compiling the source anyway, so in this sense they are
equivalent to the second-order macros used in \liquid.

The C programming language has a rich history in system programming--
specifically targeting embedded applications--and is the basis behind many
well-known projects including the linux kernel and the python programming
language.
Having said this, high-level frameworks and graphical interfaces are much more
suited to be written in C++ and will beat an implementation in C any day,
but lie far outside the scope of this project.

\subsection{Data Types}
The majority of signal processing for SDR is performed at complex baseband.
Complex numbers are handled in \liquid\ by defining data type
{\tt liquid\_float\_complex} which is binary compatible with the standard
C math type {\tt float complex} and C++ type {\tt std::complex<float>}.

Fixed-point data types are defined in the \liquidfpm\ library (see XXX).

\subsection{Building/Linking with C++}
Although \liquid\ is written in C, it can be seamlessly compiled and linked
with C++ source files.
Here is an example:
\input{listings/nco.c++.tex}

%\section{Complex Baseband}
%\label{section:complex_baseband}




%%%%%%%%%%%%%%%%%%%%%%%%%%%%%%%%%%%%%%%%%%%%%%%%%%%%%%%%%%%%%%%%%%%%%
%
%             Tutorials
%
%%%%%%%%%%%%%%%%%%%%%%%%%%%%%%%%%%%%%%%%%%%%%%%%%%%%%%%%%%%%%%%%%%%%%
\newpage
\part{Tutorials}
\label{part:tutorials}

\bigskip
\noindent
To get you started with using \liquid\ signal processing library
this manual begins with tutorials rather than diving into the details of
each signal processing module.

% TODO throw up some fancy graphic here

% 
% TUTORIAL : pll
%

\newpage
\section{Tutorial: Phase-Locked Loop}
\label{tutorial:pll}
This tutorial demonstrates the functionality of a carrier phase-locked
loop and introduces the {\tt iirfilt} object.
%
You will need on your local machine:
\begin{itemize}
\item the \liquid\ DSP libraries built and installed (see
      \S\ref{section:installation})
\item a text editor such as {\tt vim} \cite{vim:web}
\item a C compiler such as {\tt gcc} \cite{gcc:web}
\item a terminal
\end{itemize}
%
The problem statement and a brief theoretical description of
phase-locked loops is given in the next section.
A walk-through of the source code follows.

\subsection{Problem Statement}
\label{tutorial:pll:problem}
Wireless communications systems up-convert the data signal with
a high-frequency carrier before transmitting over the air.
This transmitted signal is orthogonal to other signals so long as
their bandwidths don't overlap and can be recovered at the receiver by
mixing it back down to baseband.
%This is accomplished by mixing the complex baseband signal...
%with a complex sinusoid, viz
%\[
%    s(t) = \Re\Bigl\{ m(t)\exp\{j\omega_ct\} \Bigr\}
%\]
Many digital communications systems modulate information in the phase of
the carrier requiring the receiver to demodulate the signal coherently
in order to recover the original data message.
In this regard the receiver must synchronize its carrier oscillator to
that of the transmitter.
To put it simply, the receiver must lock on to the phase of
transmitter's carrier.
One of the key advantages to performing signal processing in software is
that the radio can operate at complex baseband.
% TODO : expand on this
% (see \S\ref{section:complex_baseband}).

In this simulation, the received signal is simply a complex sinusoid
with an unknown initial carrier phase and frequency.
The carrier holds no information-bearing symbols and is simply a tone
whose frequency and phase represent the residual mismatch between the
transmitter and receiver.
The received signal $x$ at time step $k$ can be described as
%
\begin{equation}
\label{eqn:tutoral:pll:x}
    x_k = \exp\bigl\{ j(\theta + k\omega) \bigr\}
\end{equation}
%
where $j \triangleq \sqrt{-1}$ and
$\theta$ and $\omega$ represent the unknown initial carrier phase and
frequency offsets, respectively.
The receiver generates a complex sinusoid with a phase $\phi_k$ as the
phase difference between $x_k$ and $y_k$ and can be computed as
%
\begin{equation}
\label{eqn:tutoral:pll:y}
    y_k = \exp\bigl\{j\phi_k\bigr\}
\end{equation}
%
%which tries to minimize the difference between the phase of the ...
The phase error at time step $k$ is expressed as
%
\begin{equation}
\label{eqn:tutoral:pll:dphi}
    \Delta\phi_k = \arg\bigl\{ x_k y_k^* \bigr\}
\end{equation}
%
where $(^*)$ denotes complex conjugation.%
\footnote{
    Those who are savvy with communications techniques will
    appreciate that we are dealing in complex baseband and can easily
    compute the phase error estimate simply as the argument of the
    product of $x_k$ and $y_k$.
    Conventional PLLs which have operated strictly in the real domain
    multiply only the real components of $x_k$ and $y_k$ for a phase
    error estimate, assume that the loop filter rejects the
    high-frequency component, and make the approximation
    $\Delta\phi \approx \sin(\Delta\phi) = \sin(\phi-\hat{\phi})$
    for small phase errors.}
The goal of the receiver is to control $\phi_k$
(the phase of the output signal $y$ at time $k$)
to lock onto the input phase of $x$,
hence the name ``phase-locked loop.''
If the phase of the output sample $y_k$ is behind that of the input
($\Delta\phi > 0$) then $\phi$ needs to be advanced appropriately for
the next time step.
Conversely, if the phase of $y_k$ is ahead of the phase of $x_k$
($\Delta\phi < 0$) then the receiver need to retard $\phi$.

Without going into a great amount of detail, this control is
accomplished using a special filter within the loop.
This filter, known as a ``loop filter,'' is designed to reject
high-frequency noise and is described with the transfer function $H(z)$.
Specifically $H(z)$ is a 2$^{nd}$-order integrating low-pass recursive
filter with
a natural frequency $\omega_n$,
a damping factor $\zeta$, and
a loop gain $K$.
The natural frequency is the resonant frequency of $H(z)$ and for all
practical purposes is the filter's bandwidth.
Increasing $\omega_n$ permits the loop to track to the input signal
faster (reduces lock time), but also increases the amount of noise
passed through the loop.
Decreasing $\omega_n$ reduces this noise but also increases the loop's
acquisition time.
The damping factor $\zeta$ controls the stability of the filter and is
typically set to a value near $1/\sqrt{2} \approx 0.707$.
The loop gain $K$ is typically very large
(on the order of $1000$ or so).
For more detailed information on loop filter design the interested
reader is referred to \S\ref{module:nco:pll}.
% TODO : reference iirdes_pll_...() as well?

The estimated phase error $\Delta\phi_k$ is filtered using $H(z)$
resulting in an output phase estimate $\phi_{k+1}$
which is used for the subsequent output sample $y_{k+1}$ as
%
\begin{equation}
\label{eqn:tutoral:pll:y1}
    y_{k+1} = \exp\bigl\{ j\phi_{k+1} \bigr\}
\end{equation}
%
% ALGORITHM : phase-locked loop
\begin{algorithm}[H]
\caption{Phase-locked Loop Control}
\label{alg:tutorial:pll}
%\algsetup{linenosize=\footnotesize}
%\algsetup{linenodelimiter=:}
\algsetup{indent=2em}
\begin{algorithmic}[1]
\STATE $\vec{x} \leftarrow \{x_0,x_1,x_2,\ldots\}$    \COMMENT{input array}
\STATE $\hat{\phi}_0 \leftarrow 0$          \COMMENT{initial output phase}

\FOR{$k=0,\,1,\,2,\,\ldots$}
    \STATE $y_k \leftarrow \exp\bigl\{ j\hat{\phi}_k \bigr\}$ \COMMENT{compute output sample}
    \STATE $\Delta\phi_k \leftarrow \arg\bigl\{ x_k y_k^* \bigr\}$ \COMMENT{phase detector}
    \STATE $\hat{\phi}_{k+1} \leftarrow \text{filter}(\Delta\phi_k)$ \COMMENT{update output phase estimate}
\ENDFOR
\end{algorithmic}
\end{algorithm}
%
A summary of the algorithm is given in Algorithm~\ref{alg:tutorial:pll}.
In the next section we will create a simple C program to simulate a
phase-locked loop with \liquid.


\subsection{Setting up the Environment}
\label{tutorial:pll:environment}

For this tutorial and others, I assume that you are using the GNU
compiler collection ({\tt gcc}) for compiling source and linking objects
\cite{gcc:web},
and that you have a familiarity with the C (or C++) programming
language.
Create a new file {\tt pll.c} and open it with your favorite editor.
Include the headers {\tt stdio.h}, {\tt complex.h}, {\tt math.h}, and
{\tt liquid/liquid.h} and add the {\tt int main()} definition
so that your program looks like this:
%
\input{tutorials/pll_init_tutorial.c.tex}
%
Compile and link the program using {\tt gcc}:
%
\begin{Verbatim}[fontsize=\small]
    $ gcc -Wall -o pll -lm -lc -lliquid pll.c
\end{Verbatim}
%
The flag ``{\tt -Wall}'' tells the compiler to print all warnings
(unused and uninitialized variables, etc.),
``{\tt -o pll}'' specifies the name of the output program is
``{\tt pll}'', and
``{\tt -lm -lc -lliquid}'' tells the linker to link the binary against
the math, standard C, and \liquid\ DSP libraries, respectively.
Notice that the above command invokes both the compiler and the linker
collectively.
%While it is usually preferred to build an intermediate object...
%
If the compiler did not give any errors, the output executable {\tt pll}
is created which can be run as
\begin{Verbatim}[fontsize=\small]
    $ ./pll
\end{Verbatim}
%
and should simply print ``{\tt done.}'' to the screen.
You are now ready to add functionality to your program.

We will now edit the file to set up the basic simulation but without
controlling the phase of the output sinusoid.
As such the output won't track to the input resulting in a significant
amount of phase error.
This simulation will operate one sample at a time and is organized into
three sections.
First, set up the simulation parameters: the initial phase and frequency
offsets ({\tt float}),
and number of samples to run ({\tt unsigned int}).
Next, initialize the complex input and output variables
({\tt x} and {\tt y}) to zero,
as well as the state of the phase error ({\tt phase\_error})
and output phase ({\tt phi\_hat}) estimates.
Finally, set up the computational loop which generates the input and
output samples, computes the phase error between them, and then prints
the results to the screen.
%
Edit {\tt pll.c} to set up the basic simulation:
%
\input{tutorials/pll_basic_tutorial.c.tex}
%
% DISCECTION:
The variables {\tt x} and {\tt y} are of type {\tt float complex} which
contains both real and imaginary components of type {\tt float}.
%\footnote{
%    If, for some reason, you prefer C++ over C you could use the type
%    {\tt std::complex<float>} instead.
%    See \S\ref{xxx} for details.}
The function {\tt cexpf()} computes the complex exponential of its
argument which for a purely imaginary input $j\alpha$ is simply
$e^{j\alpha} = \cos\alpha + j\sin\alpha$.
% TODO : finish explanation

%
Compile and run the program as before.
The program should now output something like this:
%
\begin{Verbatim}[fontsize=\small]
      0 : phase =   0.00000000, error =   0.80000001
      1 : phase =   0.00000000, error =   0.81000000
      2 : phase =   0.00000000, error =   0.81999999
      3 : phase =   0.00000000, error =   0.82999998
      4 : phase =   0.00000000, error =   0.84000003
            ...
     35 : phase =   0.00000000, error =   1.14999998
     36 : phase =   0.00000000, error =   1.15999997
     37 : phase =   0.00000000, error =   1.17000008
     38 : phase =   0.00000000, error =   1.18000007
     39 : phase =   0.00000000, error =   1.19000006
    done.
\end{Verbatim}
%
Notice that because we aren't controlling the output phase yet
the error increases with the input phase.
In the next section we will design the loop filter to adjust the output
phase to lock onto the input signal given the phase error.

\subsection{Designing the Loop Filter}
\label{tutorial:pll:design}

Our program so far has not used any of the \liquid\ DSP libraries for
computation and has only relied on the standard C libraries for dealing
with complex math operations.
In this section we will introduce \liquid's {\tt iirfilt\_rrrf} object
to realize a recursive (infinite impulse response) filter with real
inputs, coefficients, and outputs.
Additionally we will use the function {\tt iirdes\_pll\_active\_lag()}
to design the coefficients for the PLL's filter,
specifically an ``active lag'' design.
While the explanation in this section is fairly long, relax!
We will only need to add about 15 lines of code to our program.
If you are eager to edit your program you may skip to
\S\ref{tutorial:pll:completed}.

Digital representations of infinite impulse response (IIR) filters have
two sets of coefficients: feedback and feedforward.
In the digital domain the transfer function is a ratio of the
polynomials in $z^{-1}$ where the feedforward coefficients
$\vec{b} = \{b_0, b_1, b_2, \ldots, b_{N-1}\}$
are in the numerator and the feedback coefficients
$\vec{a} = \{a_0, a_1, a_2, \ldots, a_{M-1}\}$
are in the denominator.
Specifically, the transfer function is
%
\begin{equation}
    H(z) =
        \frac{
            b_0 + b_1 z^{-1} + b_2 z^{-2} + \ldots + b_{N-1}z^{-(N-1)}
        }{
            a_0 + a_1 z^{-1} + a_2 z^{-2} + \ldots + a_{M-1}z^{-(M-1)}
        }
\end{equation}
%
This transfer function means that the output of the filter is the linear
combination of the $N$ previous filter inputs
%($\vec{x} = \{x_0, x_1, x_2, \ldots, x_{N-1}\}$)
($\vec{x}$)
and $M-1$ previous filter outputs
%($\vec{y} = \{y_0, y_1, y_2, \ldots, y_{M-1}\}$),
($\vec{y}$),
viz
%
\begin{eqnarray}
%    y_{k} =
%        \frac{1}{a_0}
%        \Bigl(
%            b_0 x_k &+& b_1 x_{k-1} + \cdots + b_{N-1} x_{k-N}\\
%                    &-& a_1 y_{k-1} - \cdots - a_{M-1} y_{k-M}
%        \Bigr)
    y[k] =
        \frac{1}{a_0}
        \Bigl(
            b_0 x[k] &+& b_1 x[k-1] + \cdots + b_{N-1} x[k-N]\\
                     &-& a_1 y[k-1] - \cdots - a_{M-1} y[k-M]
        \Bigr)
\end{eqnarray}
%
Typically the number of feedback and feedforward coefficients are equal
($M=N$), and the coefficients themselves are normalized so that $a_0=1$.

\liquid\ implements IIR filters with the {\tt iirfilt\_xxxt} family of
objects where ``{\tt xxxt}'' denotes the type definition
(see \S\ref{section:data_structures} for details).
In our example we will be using the {\tt iirfilt\_rrrf} object which
indicates that this is an IIR filter with real inputs, outputs, and
coefficients with precision of type {\tt float}.
The IIR filter objects in \liquid\ maintain their state
internally, storing the previous inputs and outputs in its internal
buffers.
Nearly every object in \liquid\ (filter or otherwise) has at least four
basic methods:
{\tt create()},
{\tt print()},
{\tt execute()}, and
{\tt destroy()}.
For our program we will need to create the filter object by passing to
it a vector of each the feedback and feedforward coefficients.
The infinite impulse response (IIR) filter we are designing is of order
two which means that $\vec{a}$ and $\vec{b}$ have three coefficients
each.

Generating the loop filter coefficients is fairly straightforward.
As stated before, the loop filter has parameters for
natural frequency $\omega_n$,
damping factor $\zeta$, and
loop gain $K$.
Furthermore the filter is 2$^{nd}$-order which means that it has three
coefficients each for $\vec{a}$ and $\vec{b}$.
\liquid\ provides a method for computing such a filter with the
{\tt iirdes\_pll\_active\_lag()} function
which accepts $\omega_n$, $\zeta$, and $K$ as inputs and generates the
coefficients in two output arrays.
The coefficients can be computed as follows:
%
\begin{Verbatim}[fontsize=\small]
    float wn = 0.1f;        // pll bandwidth
    float zeta = 0.707f;    // pll damping factor
    float K = 1000.0f;      // pll loop gain
    float b[3];             // feedforward coefficients array
    float a[3];             // feedback coefficients array
    iirdes_pll_active_lag(wn, zeta, K, b, a);
\end{Verbatim}
%
The life cycle of the IIR filter can be summarized as follows
%
\begin{Verbatim}[fontsize=\small]
    iirfilt_rrrf loopfilter = iirfilt_rrrf_create(b,3,a,3);
    float sample_in = 0.0f;
    float sample_out;
    {
        // repeat as necessary
        iirfilt_rrrf_execute(loopfilter, sample_in, &sample_out);
    }
    iirfilt_rrrf_destroy(loopfilter);
\end{Verbatim}
%
noting that the {\tt execute()} method can be repeated as many times as
necessary before the object is destroyed.

Using the code snippets above, modify your program to include the loop
filter to adjust the output signal's phase.
The input to the filter will be the {\tt phase\_error} variable, and its
output will be {\tt phi\_hat}.
Don't forget to destroy your filter object once the loop has finished
running.

\subsection{Final Program}
\label{tutorial:pll:completed}

The final program is listed below,
and a copy of the source is located in the {\tt doc/tutorials/}
subdirectory.
%
\input{tutorials/pll_tutorial.c.tex}
%
Compile the program as before, creating the executable ``{\tt pll}.''
Running the program should produce an output similar to this:
\begin{Verbatim}[fontsize=\small]
    iir filter [normal]:
      b :  0.32277358  0.07999840 -0.24277516
      a :  1.00000000 -1.99995995  0.99996001
      0 : phase =   0.25821885, error =   0.80000001
      1 : phase =   0.75852644, error =   0.55178112
      2 : phase =   1.12857747, error =   0.06147351
      3 : phase =   1.27319980, error =  -0.29857749
      4 : phase =   1.23918116, error =  -0.43319979
            ...
     35 : phase =   1.15999877, error =   0.00000751
     36 : phase =   1.17000139, error =   0.00000122
     37 : phase =   1.18000150, error =  -0.00000131
     38 : phase =   1.19000030, error =  -0.00000140
     39 : phase =   1.19999886, error =  -0.00000024
    done.
\end{Verbatim}
%
Notice that the phase error at the end of the output is very small.
The initial error (at $k=0$) is 0.8 which is the value of the
{\tt phase\_offset} parameter at the beginning of the program.
Notice also that the difference in phase of the last several samples
(i.e. the difference between the phase at steps {\tt 38} and {\tt 39})
is approximately 0.1 which is the initial frequency offset that was
given in the beginning.
Play around with the input parameters, particularly the frequency offset
and the phase-locked loop bandwidth.
Increasing the PLL bandwidth ({\tt wn}) should reduce the resulting
phase error more quickly.
The downside of having a PLL with a large bandwidth is that when the
input signal has been corrupted by noise then the phase error estimate
is also noisy.
In this tutorial no noise term was introduced. % TODO : explain more!


%% 
% TUTORIAL : modem
%

\newpage
\section{Tutorial: Digital Modem}
\label{tutorial:modem}

% 
% TUTORIAL : fec
%

\newpage
\section{Tutorial: Forward Error Correction}
\label{tutorial:fec}

This tutorial will demonstrate computation at the byte level by
introducing the forward error-correction (FEC) coding module.
Please note that \liquid\ only provides some very basic FEC
capabilities including some Hamming block codes and repeat codes.
While these codes are very fast and enough to get started,
they are not very efficient and add a lot of redunancy without providing
a strong level of correcting capabilities.
\liquid\ will use the convolutional and Reed-Solomon codes described in
{\em libfec} \cite{libfec:web} if installed on your machine.
%While not a requirement...
% maybe in the future these can be imported into liquid-dsp...

\subsection{Problem Statement}
\label{tutorial:fec:problem}
Digital communications...

\subsection{Setting up the Environment}
\label{tutorial:fec:environment}

\subsection{Final Program}
\label{tutorial:fec:completed}

The final program is listed below,
and a copy of the source is located in the {\tt doc/tutorials/}
subdirectory.
%
\input{tutorials/fec_tutorial.c.tex}
%




%%%%%%%%%%%%%%%%%%%%%%%%%%%%%%%%%%%%%%%%%%%%%%%%%%%%%%%%%%%%%%%%%%%%%
%
%             Modules
%
%%%%%%%%%%%%%%%%%%%%%%%%%%%%%%%%%%%%%%%%%%%%%%%%%%%%%%%%%%%%%%%%%%%%%
\newpage
\part{Modules}
\label{part:modules}

\bigskip
\noindent
Source code for \liquid\ is organized into {\em modules} which are, for
the most part, self-contained elements...

% TODO throw up some fancy graphic here


% 
% MODULE : agc (automatic gain control)
%

\section{agc (automatic gain control)}
\label{module:agc}

Basic usage:
\input{listings/agc.example.c.tex}

%% 
% MODULE : ann (artificial neural networks)
%

\section{ann (artificial neural networks)}
multi-layer perceptron networks, maxnets, [radial basis functions], etc.

\subsection{Multi-layer perceptron network}

Notation:
\begin{itemize}
    \item[$w$] weight
    \item[$\Delta w$] weight correction
    \item[$\phi(\cdot)$] activation function
    \item[$\phi'(\cdot)$] activation function derivative
    \item[$x$] neuron input
    \item[$v$] neuron output (before activation function)
    \item[$y$] neuron output (after activation function)
    \item[$\delta$] neuron input(output) error
\end{itemize}
Additionally, subscripts $i$ and $j$ represent the weight and node indices,
respectively, superscript $k$ denotes the layer, and $n$ represents the time.
For example, $w_{i,j}^{(k)}[n]$ is the $i^{th}$ weight of the $j^{th}$ node in
layer $k$ at time $n$.
Each neuron has an input vector $\vec{x}^{(k)}[n]$...
the output is therefore
\[
    y_j^{(k)}[n] = \phi\left( \sum_{i}{y_{i}^{(k-1)}[n] w_{i,j}^{(k)}[n]} + b_{j}^{(k)} \right)
\]

% 
% MODULE : audio
%

\newpage
\section{audio}
\label{module:audio}
The audio module in \liquid\ provides several objects and functions for
compressing, digitizing, and manipulating audio signals.
This is particularly useful for encoding audio data for wireless
communications.

\subsection{{\tt cvsd} (continuously variable slope delta)}
\label{module:audio:cvsd}
Continuously variable slope delta (CVSD) source encoding is used for data
compression of audio signals.
CVSD is a lossy compression whose quality is directly related to the sampling
frequency and is generally most practical for speech applications.
It is a form of delta modulation where $\Delta$ (the step size) is changed
continuously to minimize slope-overload distortion \cite[p. 131]{Proakis:2001}.
The output bit stream has a rate equal to that of the sampling frequency.
It is considered to be a moderate compromise between quality and complexity.

\subsubsection{Theory}
\label{module:audio:cvsd:theory}
The algorithm attempts to dynamically adjust the value of $\Delta$
to track to the input signal.
As with regular delta modulation algorithms,
if the decoded reference signal exceeds the input (the error signal is
negative), a binary {\tt 0} is sent and $\Delta$ is subtracted from the
reference, otherwise a binary {\tt 1} is sent and $\Delta$ is added.
However CVSD observes the previous $N$ transmitted bits are stored in a
buffer $\hat{\vec{b}}$;
$\Delta$ is increased by $\zeta$ if they are equal and decreased
otherwise.
This improves the dynamic range of the encoder over fixed-delta
modulation encoders.
%
A summary of the encoding procedure can be found in
Algorithm~\ref{alg:module:audio:cvsd:encoder}.
%
% ALGORITHM : CVSD encoder
\begin{algorithm}[H]
\caption{CVSD encoder algorithm}
\label{alg:module:audio:cvsd:encoder}
\algsetup{linenosize=\footnotesize}
\algsetup{linenodelimiter=:}
\algsetup{indent=2em}
\begin{algorithmic}[1]
\STATE $\vec{x} \leftarrow \{x_0,x_1,x_2,\ldots\}$  \COMMENT{input audio samples}
\STATE $v_0 \leftarrow 0$                           \COMMENT{initial output reference}
\STATE $\Delta_0 \leftarrow \Delta_{min}$           \COMMENT{initialize step size}
\STATE $\hat{\vec{b}}_0 \leftarrow \{0,0,\ldots,0\}$\COMMENT{initialize $N$-bit buffer}

\FOR{$k=0,\,1,\,2,\,\ldots$}
    \STATE $b_k \leftarrow \begin{cases} 0 & v_k  > x_k \\ 1 & \text{else}\end{cases}$
        \COMMENT{compute output bit}
    \STATE $\hat{\vec{b}}_k \leftarrow  \{\hat{b}_1,\hat{b}_2,\ldots,\hat{b}_{N-1},b_k\}$
        \COMMENT{append output bit to end of buffer}
    \STATE $m \leftarrow \sum_{i=0}^{N-1}{\hat{\vec{b}}_i}$
        \COMMENT{compute sum of last $N$ bits}
    \STATE $\Delta_k \leftarrow \begin{cases}\Delta_{k-1}\zeta & m = 0, m=N \\ \Delta_{k-1}/\zeta & \text{else}\end{cases}$
        \COMMENT{adjust step size}
    \STATE $v_{k+1} \leftarrow v_k + (-1)^{1-b_k} \Delta_k$
        \COMMENT{adjust reference value}
\ENDFOR
\end{algorithmic}
\end{algorithm}
%

The decoder reverses this process;
by retaining the past $N$ bit inputs in a buffer $\hat{\vec{b}}$,
the value of $\Delta$ can be adjusted appropriately.
A summary of the decoding procedure can be found in
Algorithm~\ref{alg:module:audio:cvsd:decoder}.
%
% ALGORITHM : CVSD encoder
\begin{algorithm}[H]
\caption{CVSD decoder algorithm}
\label{alg:module:audio:cvsd:decoder}
\algsetup{linenosize=\footnotesize}
\algsetup{linenodelimiter=:}
\algsetup{indent=2em}
\begin{algorithmic}[1]
\STATE $\vec{b} \leftarrow \{b_0,b_1,b_2,\ldots\}$  \COMMENT{input bit samples}
\STATE $v_0 \leftarrow 0$                           \COMMENT{initial output reference}
\STATE $\Delta_0 \leftarrow \Delta_{min}$           \COMMENT{initialize step size}
\STATE $\hat{\vec{b}}_0 \leftarrow \{0,0,\ldots,0\}$\COMMENT{initialize $N$-bit buffer}

\FOR{$k=0,\,1,\,2,\,\ldots$}
    \STATE $\hat{\vec{b}}_k \leftarrow  \{\hat{b}_1,\hat{b}_2,\ldots,\hat{b}_{N-1},b_k\}$
        \COMMENT{append output bit to end of buffer}
    \STATE $m \leftarrow \sum_{i=0}^{N-1}{\hat{\vec{b}}_i}$
        \COMMENT{compute sum of last $N$ bits}
    \STATE $\Delta_k \leftarrow \begin{cases}\Delta_{k-1}\zeta & m = 0, m=N \\ \Delta_{k-1}/\zeta & \text{else}\end{cases}$
        \COMMENT{adjust step size}
    \STATE $v_{k+1} \leftarrow v_k + (-1)^{1-b_k} \Delta_k$
        \COMMENT{adjust reference value}
    \STATE $y_k \leftarrow v_k$
        \COMMENT{set output value}
\ENDFOR
\end{algorithmic}
\end{algorithm}
%

%
\subsubsection{Pre-/Post-Filtering}
\label{module:audio:cvsd:filtering}
To preserve the signal's integrity the encoder applies a pre-filter
to emphasize the high-frequency information of the signal before the
encoding process.
The pre-filter is a simple 2-tap FIR filter defined as
%
\begin{equation}
    H_{pre}(z) = 1 - \alpha z^{-1}
\end{equation}
%
where $\alpha$ controls the amount of emphasis applied.
Typical values fore pre-emphasis are $0.92 < \alpha < 0.98$;
setting $\alpha=0$ completely disables this emphasis.
%
This process is reversed on the decoder by applying the inverse of
$H_{pre}(z)$ as a low-pass de-emphasis filter:
%
\begin{equation}
    H_{pre}^{-1}(z) = 
        \frac{ 1 }{ 1 - \alpha z^{-1} }
\end{equation}
%
Additionally, the decoder adds a DC-blocking filter to reject any
residual offset caused by the decoding process.
By itself the DC-blocking filter has a transfer function
%
\begin{equation}
    H_{0}(z) = 
        \frac{ 1 - z^{-1} }{ 1 - \beta z^{-1} }
\end{equation}
%
where $\beta$ controls the cut-off frequency of the filter and is
typically set very close to 1.
The default value for $\beta$ in \liquid\ is 0.99.
The full post-emphasis filter is therefore
%
\begin{equation}
    H_{post}(z) = 
    H_{pre}^{-1}(z) H_0(z) =
        \frac{
            1 - z^{-1}
        }{
            1 - (\alpha + \beta) z^{-1} + \alpha\beta z^{-2}
        }
\end{equation}
%

\subsubsection{Interface}
\label{module:audio:cvsd:interface}


The {\tt cvsd} object in \liquid\ allows the user to select both $\zeta$
as well as $N$, the number of repeated bits observed before $\Delta$ is
updated.
The combination of these values with the sampling rate yields a speech
compression algorithm with moderate quality.
Listed below is the full interface to the {\tt cvsd} object:
%
\begin{description}
\item[{\tt cvsd\_create(N,zeta,alpha)}]
    creates an {\tt agc} object with parameters $N$, $\zeta$, and
    $\alpha$.
\item[{\tt cvsd\_destroy(q)}]
    destroys a {\tt cvsd} object, freeing all internally-allocated
    memory and objects.
\item[{\tt cvsd\_print(q)}]
    prints the {\tt cvsd} object's internal parameters to the standard
    output.
\item[{\tt cvsd\_encode(q,sample)}]
    encodes a single audio sample, returning the encoded bit.
\item[{\tt cvsd\_decode(q,bit)}]
    decodes and returns a single audio sample from an input bit.
\item[{\tt cvsd\_encode8(q,samples,byte)}]
    encodes a block of 8 samples returning the result in a single byte.
\item[{\tt cvsd\_decode8(q,byte,samples)}]
    decodes a block of 8 samples from an encoded byte.
\end{description}

\subsubsection{Example}
\label{module:audio:cvsd:example}

Here is a basic example of the {\tt cvsd} object in \liquid:
%
\input{listings/cvsd.example.c.tex}
%
A demonstration of the algorithm can be seen in
Figure~\ref{fig:module:audio:cvsd} where the encoder attempts to track to an
input sinusoid.
Notice that the encoder sometimes overshoots the reference signal.
This distortion results in degradations, particularly in the upper frequency
bands.
%
\begin{figure}
\centering
  \includegraphics[trim = 0mm 0mm 0mm 0mm, clip, width=13cm]{figures.gen/audio_cvsd}
\caption{
    {\tt cvsd} example encoding a windowed sum of sine functions
    with $\zeta=1.5$, $N=2$, and $\alpha=0.95$.}
\label{fig:module:audio:cvsd}
\end{figure}
%
A more detailed example is given in
{\tt examples/cvsd\_example.c}
under the main \liquid\ project directory.

% 
% MODULE : buffer
%
\section{buffer}
\label{module:buffer}
The buffer module includes objects for storing, retrieving, and interfacing
with buffered data samples.

%
% gport
%
\subsection{gport}
\label{module:buffer:gport}
The {\tt gport} object implements a generic port to share data between
asynchronous threads.
The port itself is really just a circular (ring) buffer containing a
mutually-exclusive locking mechanism to allow processes running on independent
threads to access its data.
Because no other modules rely on the {\tt gport} object and because it
requires the pthread library, it is likely to be removed from \liquid\ in the
near future and likely put into another library, e.g. {\em liquid-threads}.

There are two ways to access the data in the {\tt gport} object: direct memory
access and indirect (copied) memory acess, each with distinct advantages and
distadvantages.
Regardless of which interface you use, the model is equivalent:
a buffer of data (initially empty) is created.
The {\it producer} is the method in charge of writing to the buffer (or
``producing'' the data).
The {\it consumer} is the method in charge of reading the data from the buffer
(or ``consuming'' it).
The producer and consumer methods can exist on completely separate threads,
and do not need to be externally synchronized.
The {\tt gport} object synchronizes the data between the ports.

\subsubsection{Direct Memory Access}
Using {\tt gport} via direct memory access is a multi-step process, equivalent
for both the producer and consumer threads.
For the sake of simplicity, we will describe the process for writing data to
the port on the producer side; the consumer process is identical.
%
\begin{enumerate}
\item the producer requests a lock on the port of a certain number of samples.
\item once the request is serviced, the port returns a pointer to an array of
      data allocated internally by the port itself.
\item the producer writes its data at this location, not exceeding the
      original number of samples requested.
\item the producer then unlocks the port, indicating how many samples were
      actually written to the buffer.
      This allows the consumer thread to read data from the buffer.
\item this process is repeated as necessary.
\end{enumerate}
%
Listed below is a minimal example demonstrating the direct memory access
method for the {\tt gport} object.
%
\input{listings/gport.direct.example.c.tex}
%

\subsubsection{Indirect/Copied Memory Access}
Indirect (or ``copied'') memory access appears similar...

\input{listings/gport.indirect.example.c.tex}

\subsubsection{Key differences between memory access modes}
While the direct memory access method provides a simpler interface--in the
sense that no external buffers are required--the user must take care in not
writing outside the bounds of the memory requested.
That is, if 256 samples are locked, only 256 values are available.
Writing more data will produce unexpected results, and could likely result in
a segmentation fault.
Furthermore, the buffer must wait until the entire requested block is
available before returning.
This could potentially increase the amount of time that each process is
waiting on the port.
Additionally, if one requests too many samples on both the producer and
consumer sides, the port could wait forever.
For example, assume one initially creates a {\tt gport} with 100 elements and
the producer initially writes 30 samples.
Immediately following, the consumer requests a lock for 100 samples which
isn't serviced because only 30 are available.
Following that, the producer requests a lock for 100 samples which isn't
serviced because only 70 are available.
This is a deadlock condition where both threads are waiting for data, and
neither request will be serviced.
The solution to this problem is actually fairly simple; the port should be
initially created as the sum of maximum size of the producer's and consumer's
requests.
That is, if the producer will at most ever request a lock on 50 samples and
the consumer will at most request a lock of 70 samples, then the port should
be initially created with a buffer size of 120.
This guarantees that the deadlock condition will never occur.

Alternatively one may use the indirect memory access method which guarantees
that the deadlock condition will never occur, even if the buffer size is 1 and
the producer writes 1000 samples while the consumer reads 1000.
This is because both the internal producer and consumer methods will write
the data as it becomes available, and do not have to wait internally until an
entire block of the requested size is ready.
This is the benefit of using the indirect memory access interface of the
{\tt gport} object.
Indirect memory access, however, requires the use of memory allocated
externally to the port.

It is important to stay consistent with the memory access mode used within a
thread, however mixed memory access modes can be used between threads on the
same port.
For example, the producer thread may use the direct memory access mode while
the consumer uses the indrect memory acess mode.

\subsubsection{Interface}
\label{module:buffer:gport:interface}

\begin{description}
\item[{\tt gport\_create()}]
    creates a new {\tt gport} object with an internal buffer of a certain
    length.
\item[{\tt gport\_destroy()}]
    destroys a {\tt gport} object, signaling {\it end of message} to any
    connected ports.
    %calling {\tt gport\_signal\_eom()}
\item[{\tt gport\_producer\_lock()}]
    locks a requested number of samples for producing, returning a {\tt void}
    pointer to the locked data array directly.
    Invoking this method can be thought of as asking the port to allocate a
    certain number of samples for writing.
    Special care must be taken by the user not to write more elements to the
    buffer than were requested.
    This function is a blocking call and waits for the data to become
    available or an {\it end of message} signal to be received.
    The data are locked until {\tt gport\_producer\_unlock()} is invoked.
    The number of unlocked samples does not have to match but cannot exceed
    those which are locked.
\item[{\tt gport\_producer\_unlock()}]
    unlocks a requested number of samples from the port.
    Use in conjunction with {\tt gport\_producer\_lock()}.
    Invoking this method can be thought of as telling the port ``I have
    written $n$ samples to the buffer you gave me earlier; release them to the
    consumer for reading.''
    The number of samples written to the port cannot exceed the initial
    request (e.g. if you request a lock for 100 samples, you should never try
    to unlock more than 100).
    There is no internal error-checking to ensure this.
    Failure to comply could result in over-writing data internally, and
    corrupt the consumer side.
\item[{\tt gport\_produce()}]
    produces $n$ samples to the port from an external buffer.
    This method is a blocking call and waits for the requested data to become
    available or an {\it end of message} signal to be received.
\item[{\tt gport\_produce\_available()}]
    operates just like {\tt gport\_produce()} except will write as many
    samples as are available when the function is called.
    Invoking this method is like telling the buffer ``I have $n$ samples, so
    write as many as you can right now.''
    It will always wait for at least one sample to become available and blocks
    until this condition is met.
\item[{\tt gport\_consumer\_lock()}]
    locks a requested number of samples for consuming, returning a {\tt void}
    pointer to the locked data array directly.
    Invoking this method can be thought of as asking the port to wait for a
    certain number of samples to be read.
    Special care must be taken by the user not to read more elements to the
    buffer than were requested.
    This function is a blocking call and waits for enough samples to become
    available or an {\it end of message} signal to be received.
    The data will be locked until {\tt gport\_consumer\_unlock()} is invoked.
    The number of unlocked samples does not have to match but cannot exceed
    those which are locked.
\item[{\tt gport\_consumer\_unlock()}]
    unlocks a requested number of samples from the port.
    Use in conjunction with {\tt gport\_consumer\_lock()}.
    Invoking this method can be though of as telling the port ``I have read
    $n$ samples from the buffer you gave me earlier; release them to the
    producer for writing.''
    The number of samples read from the port cannot exceed the initial
    request (e.g. if you request a lock for 100 samples, you should never try
    to unlock more than 100).
\item[{\tt gport\_consume()}]
    consumes $n$ samples from the port and writes to an external buffer.
    This method is a blocking call and waits for the requested data to become
    available or an {\it end of message} signal to be received.
\item[{\tt gport\_consume\_available()}]
    operates just like {\tt gport\_consume()} except will read as many samples
    as are available when the function is called.
    Invoking this method is like telling the buffer ``I have a buffer of $n$
    samples, so write to it as many as you can right now.''
    It will always wait for at least one sample to become available and blocks
    until this condition is met.
\item[{\tt gport\_signal\_eom()}]
    signals {\it end of message} to any connected {\tt gport}.
    This tells the port to stop waiting for data (on both the producer and
    consumer side) and return.
    This method prevents lock conditions where, e.g., the producer is waiting
    for several samples to become available, but the consumer has finished its
    process.
    This method is normally invoked only during {\tt gport\_destroy()}.
\item[{\tt gport\_clear\_eom()}]
    ({\it untested})
    clears the {\it end of message} signal.
\end{description}


\subsubsection{Problem areas}
When using the direct memory access method, the size of the data request
during lock is limited by the size of the port.
[[race/lock conditions?]]

\subsection{{\tt window} buffer}
\label{module:buffer:window}
The {\tt window} object is used to implement a sliding window buffer.
It is essentially a first-in, first-out queue but with the constraint that a
fixed number of elements is always available, and the ability to read the
entire queue at once.
This is particularly useful for filtering objects which use time-domain
convolution of a fixed length to compute its outputs.
Unlike the {\tt gport} object, {\tt window} objects operate on a known data
type, e.g.
{\it float} ({\tt windowf}), and
{\it float complex} ({\tt windowcf}).
%{\it unsigned int} ({\tt uiwindow}).

The buffer has a fixed number of elements which are initially zeros.
Values may be pushed into the end of the buffer (into the ``right'' side)
using the {\tt push()} method, or written in blocks via {\tt write()}.
In both cases the oldest data samples are removed from the buffer (out of the
``left'' side).
When it is necessary to read the contents of the buffer, the {\tt read()}
method returns a pointer to its contents.
\liquid\ implements this shifting method in the same manner as a ring buffer,
and linearizes the data very efficiently, without performing any unnecessary
data memory copies.
Effectively, the window looks like:

\begin{centering}
\includegraphics[width=16cm]{figures.pgf/window}
\end{centering}

\subsubsection{Interface}
\label{module:buffer:window:interface}

\begin{description}
\item[{\tt window\_create()}]
    creates a new window of a certain length.
\item[{\tt window\_recreate()}]
    extends an existing window's size, similar to the standard C library's
    {\tt realloc()}.
    If the size of the new window is larger than the old one, the newest
    values are retained at the beginning of the buffer and the oldest values
    are truncated.
    % see \ref{listing:buffer:window}~line~23
    If the size of the new window is smaller than the old one, the oldest
    values are truncated.
    % see \ref{listing:buffer:window}~line~27
\item[{\tt window\_clear()}]
    clears the contents of the buffer by setting all internal values to zero.
\item[{\tt window\_index()}]
    retrieves the $n^{th}$ sample in the window.
    This is equivalent to first invoking {\tt read()} and then indexing on the
    resulting pointer, however the result is obtained much faster.
    Therefore invoking {\tt window\_index(w,0)} returns the {\em oldest} value
    in the window.
\item[{\tt window\_read()}]
    reads the contents of the window by returning a pointer to the aligned
    internal memory array.
    This method guarantees that the elements are linearized.
    This method should {\em only} be used for reading; writing values to the
    buffer has unknown results.
\item[{\tt window\_push()}]
    shifts a single sample into the right side of the window, pushing the
    oldest (left-most) sample out of the end.
    Unlike stacks, the {\tt window} object has no equivalent ``pop'' method,
    as values are retained in memory until they are overwritten.
\item[{\tt window\_write()}]
    writes a block of data to the window.
    Effectively, it is equivalent to pushing each sample one at a time, but
    executes much faster.
\end{description}

Here is an example demonstrating the basic functionality of the window object.
The comments show the internal state of the window after each function call as
if the window were a simple C array.
\input{listings/window.example.c.tex}


% 
% MODULE : channel
%

\newpage
\section{channel}
\label{module:channel}
communications channel modeling: additive noise, multi-path fading, etc.


% 
% MODULE : dotprod (vector dot product)
%

\section{dotprod}
\label{module:dotprod}


% 
% MODULE : equalization
%

\newpage
\section{equalization}
\label{module:equalization}
adaptive equalizers: LMS, RLS, blind...

This section describes the functionality of two digital linear adaptive
equalizers implemented...

\subsection{System Description}
\label{module:equalization:system}
Suppose a known transmitted symbol sequence
$\vec{d} = [ d(0), d(1), \ldots ,d(N-1) ]$
which passes through an unknown channel filter $\vec{h}_n$ of length
$q$.
The received symbol at time $n$ is therefore
%
\begin{equation}
    y(n) = \sum\limits_{k=0}^{q-1}{h_n(k)d(n-k)} + \varphi(n)
\end{equation}
%
where $\varphi(n)$ represents white Gauss noise.
The adaptive linear equalizer attempts to use a finite impulse response (FIR)
filter $\vec{w}$ of length $p$ to estimate the transmitted symbol, using only
the received signal vector $\vec{y}$ and the known data sequence $\vec{d}$,
viz
\[
    \hat{d}(n) = \vec{w}_n^T \vec{y}(n)
\]
where $\vec{y}_n = [ y(n), y(n-1),\ldots, y(n-p+1) ]^T$.
Several methods for estimating $\vec{w}$ are known in the literature, and
typically rely on iteratively adjusting $\vec{w}$ with each input though a
recursion algorithm.
This section provides a very brief overview of two prevalent adaptation
algorithms;
for a more in-depth discussion the interested reader is referred to
\cite{Proakis:2001,Haykin:2002}.

\subsection{{\tt eqlms} (least mean-squares equalizer)}
\label{module:equalization:eqlms}
The least mean-squares (LMS) algorithm adapts the coefficients of the filter
estimate using a steepest descent (gradient) of the instantaneous {\it a priori}
error.
The filter estimate at time $n+1$ follows the following recursion
\begin{equation}
\label{eq:lms:weight_update}
\vec{w}_{n+1} = \vec{w}_{n} - \mu \vec{g}_n
\end{equation}
where $\mu$ is the iterative step size, and
$\vec{g}_n$ the normalized gradient vector, estimated from the error signal
and the coefficients vector at time $n$.

\subsection{{\tt eqrls} (recursive least-squares equalizer)}
\label{module:equalization:eqrls}
The recursive least-squares (RLS) algorithm attempts to minimize the
time-average weighted square error of the filter output, viz
\begin{equation}
c(\vec{w}_n) = \sum\limits_{i=0}^{n}{ \lambda^{i-n} \left| d(i)-\hat{d}(i)\right|^2 }
\end{equation}
where the forgetting factor $0<\lambda\leq 1$ which introduces
exponential weighting into past data, appropriate for time-varying
channels.
The solution to minimizing the cost function $c(\vec{w}_n)$ is achieved by
setting its partial derivatives with respect to $\vec{w}_n$ equal to zero.
The solution at time $n$ involves inverting the weighted cross correlation
matrix for $\vec{y}_n$, a computationally complex task.
This step can be circumvented through the use of a recursive algorithm which
attempts to estimate the inverse using the {\it a priori} error from the
output of the filter.
The update equation is
\begin{equation}
\label{eq:rls:weight_update}
\vec{w}_{n+1} = \vec{w}_n + \Delta_{n}
\end{equation}
where the correction factor $\Delta_{n}$ depends on $\vec{y}_n$ and $\vec{w}_n$,
and involves several $p \times p$ matrix multiplications.
The RLS algorithm provides a solution which converges much faster than the LMS
algorithm, however with a significant increase in computational complexity and
memory requirements.

\subsection{Interface}
\label{module:equalization:interface}
The {\tt eqlms} and {\tt eqrls} have nearly identical interfaces, so we will
leave the discussion to the {\tt eqlms} object here.
Like most objects in \liquid, {\tt eqlms} follows the typical
{\tt create()}, {\tt execute()}, {\tt destroy()} lifecycle.
Training is accomplished either one sample at a time, or in a batch cycle.
If trained one sample at a time, the symbols must be trained in the proper
order, otherwise the algorithm won't converge.
Here is a simple example:
%
\input{listings/eqlms_cccf.example.c.tex}
%
For more detailed examples, see
{\tt examples/eqlms\_cccf\_example.c} and
{\tt examples/eqrls\_cccf\_example.c}.

\subsection{Demonstration}
\label{module:equalization:demo}
The performance of the {\tt eqlms} and {\tt eqrls} equalizers are compared by
generating a channel with an impulse response representing a strong
line-of-sight (LoS) component followed by random echoes.
Each was trained on 512 iterations of a known QPSK-modulated training sequence
with learning rate parameters $\mu=0.999$ and $\lambda=0.999$ for the LMS and
RLS algorithms, respectively.
A small amount of noise was injected after the channel filter to demonstrate
the robustness of the algorithms.
The results of two simulations are shown in
figures~\ref{fig:module:equalization:example1} and
\ref{fig:module:equalization:example2};
the first demonstrating a 10-tap equalizer applied to the response of a 6-tap
channel with an SNR of 40dB, while the second demonstrates a 28-tap equalizer
for a 12-tap channel with an SNR of just 10dB.

The passband power spectral densities (PSD) of the channel and the equalizer
outputs are depicted in figures~\ref{fig:module:equalization:example1:psd} and
\ref{fig:module:equalization:example2:psd}.
Notice that the inter-symbol interference of the channel causes its PSD to
have a non-flat response.
Theoretically, if the inter-symbol interference is completely removed, the
response of both the channel and the equalizer will be completely flat
(neglecting any noise present).
While the PSD of the equalized output is nearly flat in the figures,
it is important to realize that these algorithms minimize a cost function
defined as the square of the {\it a priori} filter output error, and do not
necessarily force the PSD to zero.
The classic zero-forcing equalizer has several drawbacks:
\begin{enumerate}
\item the equalizing filter which would give this response is not
      necessarily realizable; that is, not all channels can be
      perfectly inverted,
\item forcing the frequency response to zero increases the noise
      terms of frequencies where the spectra of the channel
      response is low.  In this regard, the zero-forcing equalizer
      only reduces inter-symbol interference and does not maximize
      the ratio of signal power to both interference {\it and}
      noise power as the LMS and RLS algorithms do.
\end{enumerate}
It is interesting to note that both the LMS and RLS equalizers converge to
nearly the same solution.
In both scenarios, however, the RLS equalizer has a slightly lower error after
training while converging to its error minimum much faster.
The RLS equalizer, however, has a much higher computational complexity.

%-------------------- FIGURE: EQUALIZER EXAMPLE 1 --------------------
\begin{figure}
\centering
\mbox{
  \subfigure[PSD] {
      \includegraphics[trim = 16mm 0mm 18mm 0mm, clip, width=6cm]{figures.gen/equalizer_example1_psd}
      \label{fig:module:equalization:example1:psd}
    } \quad
  \subfigure[constellation] {
      \includegraphics[trim = 16mm 0mm 18mm 0mm, clip, width=6cm]{figures.gen/equalizer_example1_const}
      \label{fig:module:equalization:example1:constellation}
    } \quad
}
\mbox{
  \subfigure[taps] {
      \includegraphics[trim = 23mm 0mm 23mm 0mm, clip, width=6cm]{figures.gen/equalizer_example1_taps}
      \label{fig:module:equalization:example1:taps}
    } \quad
  \subfigure[mean-squared error] {
      \includegraphics[trim = 16mm 0mm 16mm 0mm, clip, width=6cm]{figures.gen/equalizer_example1_mse}
      \label{fig:module:equalization:example1:mse}
    } \quad
}
% trim = left bottom right top
\caption{Scenario 1: 10-tap equalizer for a 6-tap channel with 40dB SNR}
\label{fig:module:equalization:example1}
\end{figure}



%-------------------- FIGURE: EQUALIZER EXAMPLE 2 --------------------
\begin{figure}
\centering
\mbox{
  \subfigure[PSD] {
      \includegraphics[trim = 16mm 0mm 18mm 0mm, clip, width=6cm]{figures.gen/equalizer_example2_psd}
      \label{fig:module:equalization:example2:psd}
    } \quad
  \subfigure[constellation] {
      \includegraphics[trim = 16mm 0mm 18mm 0mm, clip, width=6cm]{figures.gen/equalizer_example2_const}
      \label{fig:module:equalization:example2:constellation}
    } \quad
}
\mbox{
  \subfigure[taps] {
      \includegraphics[trim = 23mm 0mm 23mm 0mm, clip, width=6cm]{figures.gen/equalizer_example2_taps}
      \label{fig:module:equalization:example2:taps}
    } \quad
  \subfigure[mean-squared error] {
      \includegraphics[trim = 16mm 0mm 16mm 0mm, clip, width=6cm]{figures.gen/equalizer_example2_mse}
      \label{fig:module:equalization:example2:mse}
    } \quad
}
% trim = left bottom right top
\caption{Scenario 2: 28-tap equalizer for a 12-tap channel with 10dB SNR}
\label{fig:module:equalization:example2}
\end{figure}



% 
% MODULE : fec (forward error correction)
%

\newpage
\section{fec (forward error correction)}
\label{module:fec}
%(basic), checksum, crc, Hamming block codes...
The fec module implements a set of forward error-correction codes for
ensuring and validating data integrity through a noisy channel.
Redundant ``parity'' bits are added to a data sequence to help correct
errors introduced by the channel.
The number of correctable errors depends on the number of parity bits of the
coding scheme, which in turn affects its rate (efficiency).
The {\tt fec} object realizes forward error-correction capabilities in
\liquid\ while the methods {\tt checksum()} and {\tt crc32()} strictly
implement error detection.
Certain FEC schemes are only available to \liquid\ by installing the external
{\tt libfec} library \cite{libfec:web}, available as a free download.
A few low-rate (and fairly low efficiency) codes are availble internally.

%The {\tt packetizer} object (Section~\ref{module:framing:packetizer})
%relies on the {\tt fec} objects and {\tt crc32} functions.

\subsection{Cyclic Redundancy Check (Error Detection)}
\label{module:fec:crc}

A cyclic redundancy check (CRC) is, in essence, a strong algebraic error
detection code that computes a key on a block of data using base-2
polynomials.
While it is a strong error-detection method, a CRC is not an error-correction
code.
Here is a simple example:
%
\input{listings/crc.example.c.tex}
%
Also available for error detection in \liquid\ is a checkssum.
A checksum is a simple way to validate data received through un-reliable means
(e.g. a noisy channel).
A checksum is, in essence, a weak error detection code that simply counts the
number of ones in a block of data (modulo 256).
The limitation, however, is that multiple bit errors might result in a false
positive validation of the corrupted data.
The checksum is not a strong an error detection scheme as the cyclic
redundany check.
%
Table~\ref{tab:crc:codecs} lists the available codecs and gives a brief
description for each.
%
% ------------ TABLE: CRC CODING SCHEMES ------------
\begin{table*}
\caption{Error-detection codecs available in \liquid}
\label{tab:crc:codecs}
\centering
{\small
\begin{tabular*}{0.75\textwidth}{l@{\extracolsep{\fill}}ll}
\toprule
{\it scheme} &
{\it size (bits)} &
{\it description}\\\otoprule
%
{\tt CRC\_UNKNOWN}      & -     & unknown/unsupported scheme\\
{\tt CRC\_NONE}         & 0     & no error-detection\\
{\tt CRC\_CHECKSUM}     & 8     & basic checksum\\
{\tt CRC\_8}            & 8     &  8-bit CRC, poly={\tt 0x07}\\
{\tt CRC\_16}           & 16    & 16-bit CRC, poly={\tt 0x8005}\\
{\tt CRC\_24}           & 24    & 24-bit CRC, poly={\tt 0x5D6DCB}\\
{\tt CRC\_32}           & 32    & 32-bit CRC, poly={\tt 0x04C11DB7}\\\bottomrule

\end{tabular*}
}
\end{table*}%
% ------------------------
%
For a detailed example program, see {\tt examples/crc\_example.c} in the
main \liquid\ directory.


\subsection{{\tt h74}, {\tt h84}, {\tt h128} (Hamming codes)}
\label{module:fec:hamming}
Hamming codes are a specific type of block
code which use parity bits capable of correcting one bit error in the block.
With the addition of an extra parity bit, they are able to detect up to two
errors, but are still only able to correct one.
\liquid\ implements the Hamming(7,4), Hamming(8,4), and Hamming(12,8)
codes.
The Hamming(8,4) can detect one additional error over the Hamming(7,4)
code;
however at the time of writing this document the number of detected
errors is not passed to the user so the Hamming(8,4) code is effectively
the same as Hamming(7,4) but with a lower rate.
%
Additionally, \liquid\ implements the Hamming(12,8) code which accepts an
8-bit symbol and adds four parity bits, extending it to a 12-bit symbol.
This yields a theoretical rate of $2/3$, and actually has a performance very
similar to that of the Hamming(7,4) code, even with a higher rate.

\subsection{{\tt rep3}, {\tt rep5} (simple repeat codes)}
\label{module:fec:rep}
The {\tt rep3} code is a simple repeat code which simply repeats the message
twice (transmits it three times).
The decoder takes a majority vote of the bits received by applying a simple
series bit masks.
If the original bit is represented as $s$, then the transmitted bits are
$s s s$.
Let the received bit sequence be $r_0 r_1 r_2$.
The estimated transmitted bit is {\tt 0} if the sum of the received bits is
less than 2, and {\tt 1} otherwise.
This is equivalent to
\[
    \hat{s} =   (r_0 \land r_1) + 
                (r_0 \land r_2) + 
                (r_1 \land r_2) 
\]
where $+$ represents logical {\it or} and $\land$ represents
logical {\it and}.
An error is detected if
\[
    \hat{e} =   (r_0 \oplus r_1) + 
                (r_0 \oplus r_2) + 
                (r_1 \oplus r_2) 
\]
where $\oplus$ represents logical {\it exclusive or}.
In this fashion it is easy to decode several bytes of data at a time because
machine architectures have low-level bit-wise manipulation instructions which
can compute logical {\it exclusive or} and {\it or} very quickly.
This is precisely how \liquid\ decodes {\tt rep3} data, only in this case,
$s$, $r_0$, $r_1$, and $r_2$ represent a bytes of data rather than bits.

The {\tt rep5} code operates similarly, except that it transmits five copies
of the original data sequence, rather than just three.
The decoder takes the five received bits $r_0,\ldots,r_4$ and adds (modulo
2) the logical {\it and} of every combination of three bits, viz
\[
    \hat{s} = \sum_{i\ne j \ne k} {(r_i \land r_j \land r_k)}
\]
This roughly doubles the number of clock cycles to decode over {\tt rep3}.

It is well-known that repeat codes do not have strong error-correction
capabilities for their rate, are are located far from the Shannon capacity
bound \cite{Proakis:2001}.
They are exceptionally weak relative to convolutional Viterbi and Reed-Solomon
codes.
However, their simplicity in implementation and low computational complexity
gains them a place in digital communications, particularly in software radios
where spectral efficiency goals might be secondary to processing contraints.

\subsection{{\tt libfec} (convolutional and  Reed-Solomon codes)}
\label{module:fec:libfecv}
\liquid\ takes advantage of convolutional and Reed-Solomon codes defined in
{\tt libfec} \cite{libfec:web}.
These codes have much stronger error-correction capabilities than {\tt rep3},
{\tt rep5}, {\tt h74}, {\tt h84}, and {\tt h128}
but are also much more computationally intensive to the host processor.
\liquid\ uses the rate $1/2 (K=7)$, $1/2 (K=9)$, $1/3 (K=9)$, and
$r1/6 (K=15)$ codes defined in {\tt libfec}, but extends the two half-rate
codes to punctured codes.
These punctured codes (also known as ``perforated'' codes) are not as strong
and cannot correct as many errors, but are more efficient and use less
overhead than their half-rate counterparts.
%
The 8-bit Reed-Solomon code is a (255,223) block code, also defined in
{\tt libfec}.
Nominally, the scheme accepts 223 bytes (8-bit symbols) and adds 32 parity
symbols to form a 255-symbol encoded block.
%
{\tt libfec} is an external library that \liquid\ will leverage if
installed, but will still compile otherwise
(see Section~\ref{section:installation:building} for details).

\subsection{Interface}
\label{module:fec:interface}
In designing the {\tt fec} interface, we have tried to keep simplicity and
reconfigurability in mind.
The various forward error-correction schemes accept bits or symbols
formatted in different lengths and have vastly different interfaces.
This potentially makes switching from one scheme to another difficult as one
needs to restructure the data accordingly.
\liquid\ takes care of all this formatting under the hood; regardless of the
scheme used, the {\tt fec} object accepts a block of uncoded data bytes and
encodes them into an output block of coded data bytes.

%The goal is to allow flexibility in the allocation ...
%TODO finish interface description

\begin{description}
\item[{\tt fec\_create(scheme,*opts)}]
    creates a {\tt fec} object of a specific scheme
    (see Table~\ref{tab:fec:codecs} for available codecs).
    Notice that the length of the input message does not need to be
    specified until {\tt encode()} or {\tt decode()} is invoked.
    The second argument is intended for future development and should be
    ignored by passing the {\tt NULL} pointer
    (see example below).
\item[{\tt fec\_recreate(q,scheme,opts)}]
    recreates an existing {\tt fec} object with a different scheme.
\item[{\tt fec\_destroy(q)}]
    destroys a {\tt fec} object, freeing all internally-allocated memory
    arrays.
\item[{\tt fec\_encode(q,n,*msg\_dec,*msg\_enc)}]
    runs the error-correction encoder scheme on an $n$-byte input data
    array {\tt msg\_dec}, storing the result in the output array
    {\tt msg\_enc}.
    To obtain the length of the output array necessary, use the
    {\tt fec\_get\_enc\_msg\_length()} method.
\item[{\tt fec\_decode(q,n,*msg\_enc,*msg\_dec)}]
    runs the error-correction decoder on an input array
    {\tt msg\_enc} of $k$ encoded bytes.
    The resulting best-effort decoded message is written to the $n$-byte
    output array {\tt msg\_dec}, allocated by the user.
    Notice that like the {\tt fec\_encode()} method, the input length
    $n$ refers to the {\em decoded} message length.
    Depending upon the error-correction capabilities of the scheme, the
    resulting data might have been corrupted,
    and therefore it is recommended to use either a checksum or a
    cyclic redundancy check (section~\ref{module:fec:crc})
    to validate data integrity.
\item[{\tt fec\_get\_enc\_msg\_length(scheme,n)}]
    returns the length $k$ of the encoded message in bytes
    for an uncoded input of $n$ bytes using the specified encoding
    scheme.
    This method can be called before the {\tt fec} object is created and
    is useful for allocating initial memory arrays.
\end{description}
%
Listed below is a simple example demonstrating the basic interface to
the {\tt fec} encoder/decoder object:
%
\input{listings/fec.example.c.tex}
%
For a more detailed example demonstrating the full capabilities of the
{\tt fec} object, see {\tt examples/fec\_example.c}.
%
%\subsection{encoder/decoder options}
%\label{module:fec:codecs}
Table~\ref{tab:fec:codecs} lists the available codecs and gives a brief
description for each.
All convolutional and Reed-Solomon codes are available only if {\tt libfec} is
installed \cite{libfec:web}.

% ------------ TABLE: FEC CODING SCHEMES ------------
\begin{table*}
\caption{Forward error-correction codecs available in \liquid}
\label{tab:fec:codecs}
\centering
{\small
\begin{tabular*}{0.75\textwidth}{l@{\extracolsep{\fill}}ll}
\toprule
{\it scheme} &
{\it asymptotic rate} &
{\it description}\\\otoprule
%
{\tt FEC\_UNKNOWN}              & -         & unknown/unsupported scheme\\
{\tt FEC\_NONE}                 & 1         & no error-correction\\
{\tt FEC\_REP3}                 & 1/3       & simple repeat code\\
{\tt FEC\_REP5}                 & 1/5       & simple repeat code\\
{\tt FEC\_HAMMING74}            & 4/7       & Hamming (7,4) block code\\
{\tt FEC\_HAMMING84}            & 1/2       & Hamming (7,4) with extra parity bit\\
{\tt FEC\_HAMMING128}           & 2/3       & Hamming (12,8) block code\\\midrule
%
% codecs not defined internally (see http://www.ka9q.net/code/fec/)
{\tt FEC\_CONV\_V27}            & 1/2       & $K=7$, $d_{free}=10$\\
{\tt FEC\_CONV\_V29}            & 1/2       & $K=9$, $d_{free}=12$\\
{\tt FEC\_CONV\_V39}            & 1/3       & $K=9$, $d_{free}=18$\\
{\tt FEC\_CONV\_V615}           & 1/6       & $K=15$, $d_{free}<=57$ (Heller 1968)\\\midrule
%
% punctured (perforated) codes
{\tt FEC\_CONV\_V27P23}         & 2/3       & $K=7$, $d_{free}=6$\\
{\tt FEC\_CONV\_V27P34}         & 3/4       & $K=7$, $d_{free}=5$\\
{\tt FEC\_CONV\_V27P45}         & 4/5       & $K=7$, $d_{free}=4$\\
{\tt FEC\_CONV\_V27P56}         & 5/6       & $K=7$, $d_{free}=4$\\
{\tt FEC\_CONV\_V27P67}         & 6/7       & $K=7$, $d_{free}=3$\\
{\tt FEC\_CONV\_V27P78}         & 7/8       & $K=7$, $d_{free}=3$\\\midrule
%
{\tt FEC\_CONV\_V29P23}         & 2/3       & $K=9$, $d_{free}=7$\\
{\tt FEC\_CONV\_V29P34}         & 3/4       & $K=9$, $d_{free}=6$\\
{\tt FEC\_CONV\_V29P45}         & 4/5       & $K=9$, $d_{free}=5$\\
{\tt FEC\_CONV\_V29P56}         & 5/6       & $K=9$, $d_{free}=5$\\
{\tt FEC\_CONV\_V29P67}         & 6/7       & $K=9$, $d_{free}=4$\\
{\tt FEC\_CONV\_V29P78}         & 7/8       & $K=9$, $d_{free}=4$\\\midrule
% 
% Reed-Solomon codes
{\tt FEC\_RS\_M8}               & 223/255   & Reed-Solomon block code, $m=8$\\\bottomrule


\end{tabular*}
}
\end{table*}%
% ------------------------


%-------------------- FIGURE: FEC BER (BUILT-IN) --------------------
\begin{figure}
\centering
\mbox{
  \subfigure[BER vs. $E_s/N_0$]
  {
    \includegraphics[trim = 3mm 2mm 0mm 2mm, clip, width=14cm]{figures.gen/fec_ber_esn0_hamming}
    \label{fig:fec:hamming_ber:esn0}
  }
} \quad
\mbox{
  \subfigure[BER vs. $E_b/N_0$]
  {
    \includegraphics[trim = 3mm 2mm 0mm 2mm, clip, width=14cm]{figures.gen/fec_ber_ebn0_hamming}
    \label{fig:fec:hamming_ber:ebn0}
  } \quad
}
% trim = left bottom right top
\caption{{\tt fec} bit error rates for built-in \liquid\ codecs}
\label{fig:fec:hamming_ber}
\end{figure}



%-------------------- FIGURE: FEC BER (CONVOLUTIONAL) --------------------
\begin{figure}
\centering
\mbox{
  \subfigure[BER vs. $E_s/N_0$]
  {
    \includegraphics[trim = 3mm 2mm 0mm 2mm, clip, width=14cm]{figures.gen/fec_ber_esn0_conv}
    \label{fig:fec:conv_ber:esn0}
  }
} \quad
\mbox{
  \subfigure[BER vs. $E_b/N_0$]
  {
    \includegraphics[trim = 3mm 2mm 0mm 2mm, clip, width=14cm]{figures.gen/fec_ber_ebn0_conv}
    \label{fig:fec:conv_ber:ebn0}
  } \quad
}
% trim = left bottom right top
\caption{{\tt fec} bit error rates for convolutional codes}
\label{fig:fec:conv_ber}
\end{figure}



%-------------------- FIGURE: FEC BER (CONVOLUTIONAL, PUNCTURED) --------------------
\begin{figure}
\centering
\mbox{
  \subfigure[BER vs. $E_s/N_0$]
  {
    \includegraphics[trim = 3mm 2mm 0mm 2mm, clip, width=14cm]{figures.gen/fec_ber_esn0_convpunc}
    \label{fig:fec:convpunc_ber:esn0}
  }
} \quad
\mbox{
  \subfigure[BER vs. $E_b/N_0$]
  {
    \includegraphics[trim = 3mm 2mm 0mm 2mm, clip, width=14cm]{figures.gen/fec_ber_ebn0_convpunc}
    \label{fig:fec:convpunc_ber:ebn0}
  } \quad
}
% trim = left bottom right top
\caption{{\tt fec} bit error rates for punctured convolutional codes}
\label{fig:fec:convpunc_ber}
\end{figure}




% 
% MODULE : fft (fast Fourier transform)
%

\section{fft (fast Fourier transform)}
\label{module:fft}

Given a vector of complex time-domain samples
$\vec{x} = \left[x_0,x_1,\ldots,x_{N-1}\right]^T$\ldots

Forward transform:
\[
    X(k) = \sum_{i=0}^{N-1}{x(i) e^{-j 2 \pi k i/N}}
\]

Reverse transform:
\[
    x(n) = \sum_{i=0}^{N-1}{X(i) e^{ j 2 \pi n i/N}}
\]

\liquid\ implements only the basic decimation-in-time FFT algorithm for
radix-2 transforms, however, will use the fftw3 library [cite:fftw3.org] if
available.


% 
% MODULE : filter
%
\section{filter}
\label{module:filter}

% fir_filter
\subsection{fir\_filter}
Finite impulse response filter.
The output $y$ is the convolution of the input $x$ with the filter
coefficients (impulse response) $h$, viz
\[
    y_n = \sum_{k=0}^{N-1}{ h_k x_{N-k-1} }
\]
where $\vec{h} = \{h_0,h_1,\ldots,h_{N-1}\}$ is the filter impulse response.

% firhilb
\subsection{firhilb (finite impulse response Hilbert transform)}

% firpfb
\subsection{firpfb (finite impulse response polyphase filter bank)}

% interp
\subsection{interp (interpolator)}

% decim
\subsection{decim (decimator)}

% qmfb
\subsection{qmfb (quadrature mirror filter bank)}

% iir_filter
\subsection{iir\_filter}
Infinite impulse response filter.
\[
    y_n = \sum_{k=0}^{N-1}{ b_k x_{N-k-1} } +
          \sum_{k=0}^{N-1}{ a_k y_{N-k-1} }
\]
where $\vec{b} = \{b_0,b_1,\ldots,b_{N-1}\}$ are the feed-forward parameters
and $\vec{a} = \{a_0,a_1,\ldots,a_{N-1}\}$ are the feed-back parameters.
\[
    H(z) = \frac{\sum_{k=0}^{N-1}{b_k z^{-k}}}
                {\sum_{k=0}^{N-1}{a_k z^{-k}}}
\]
Typically, $H(z)$ is normalized such that $a_0=1$.

\subsection{resamp2 (halfband resampler)}
resamp2 is a halfband resampler used for efficient interpolation and
decimation

\subsection{resamp (arbitrary resampler)}

\subsection{symsync (symbol synchronizer)}

\subsection{symsync2 (halfband symbol synchronizer)}

\subsection{firfarrow (finite impulse response Farrow filter)}

\subsection{lf2 (second-order integrating loop filter)}



% 
% MODULE : framing
%

\section{framing}
\label{module:framing}
The framing module contains objects and methods for packaging data into
manageable frames and packets.
For convention, \liquid\ refers to a ``packet'' as a group of binary
data bytes that need to be communicated over a wireless link.
A ``frame'' is a representation of the data once it has been properly
partitioned, encoded, and modulated before transmitting over the air.
Included in this module are the {\tt packetizer}, 
{\tt frame[gen|sync]64}, and {\tt flexframe[gen|sync]} objects which
greatly simplify over-the-air digital communication of data.

\subsection{{\tt packetizer}}
\label{module:framing:packetizer}

The liquid packetizer is a structure for abstracting multi-level forward
error-correction from the user.
The packetizer accepts a buffer of uncoded data bytes and adds a 32-bit
cyclic redundancy check (crc) before applying two levels of forward error-
correction and bit-level interleaving.  The user may choose any two 
supported FEC schemes (including none) and the packetizer object will
handle buffering and data management internally, providing a truly abstract
interface.  The same is true for the packet decoder which accepts an array
of possibly corrupt data and attempts to recover the original message using
the FEC schemes provided.  The packet decoder returns the validity of the
resulting CRC as well as its best effort of decoding the message.

The packetizer also allows for re-structuring if the user wishes to change
error-correction schemes or data lengths.  This is accomplished with the
{\tt packetier\_recreate()} method.

Here is a minimal example demonstrating the packetizer's most basic
functionality:
\input{listings/packetizer.example.c.tex}

See also: fec module, {\tt examples/packetizer\_example.c}

\subsection{{\tt frame64}, {\tt flexframe} (basic framing objects)}
\label{module:framing:frame}
The {\tt framegen64} and {\tt framesync64} objects implement a basic framing
structure for communicating packetized data over the air.

The {\tt flexframegen} and {\tt flexframesync} objects are similar to their
{\tt frame[gen|sync]64} counterparts, however extend functionality to include
a number of options in structuring the frame.

Both frames consist of six basic parts:

%\begin{tabular*}{0.95\textwidth}{l@{\extracolsep{\fill}}lll}
%\toprule
%{\it name}      & {\it \# symbols}  & {\it \# bytes}& {\it description}  \\
%\otoprule
%ramp up         & 64                & -             & BPSK ramp up sequence \\
%phasing pattern & 64                & -             & BSPK preamble phasing pattern \\
%p/n sequence    & 64                & -             & BPSK p/n synchronization sequence \\
%header          & 256               & 32            & QPSK, $r=1/2$-coded header \\
%payload         & 512               & 64            & QPSK, $r=1/2$-coded payload \\
%ramp down.      & 64                & -             & ramp down sequence \\
%\bottomrule
%\end{tabular*}

\begin{description}
\item[{\sf ramp/up}]
    gracefully increases the output signal level to avoid ``key clicking'' and
    reduce spectral side-lobes in the transmitted signal.
    Furthermore, it allows the receiver's automatic gain control unit to
    lock on to the incoming signal, preventing sharp transitions in its
    output.
\item[{\sf phasing pattern}]
    is BPSK pattern which flips phase for each transmitted symbol
    ({\tt~+1,-1,+1,-1,$\ldots$}).
    This sequence serves several purposes but primarily to help the receiver's
    symbol synchronization circuit to lock onto the proper timing phase.
    [This works] because the phasing pattern maximizes the number of symbol
    transitions [reword].
\item[{\sf p/n sequence}]
    is an $m$-sequence (see section~\ref{module:sequence}) exhibiting good
    auto- and cross-correlation properties.
    %The binary sequence is modulated using BPSK so that 
    This sequence aligns the frame synchronizers to the remainder of the
    frame, telling them when to start receiving and decoding the frame header,
    as well as if the phase of the received signal needs to be reversed.
    At this point, the receiver's AGC, carrier PLL, and timing PLL should all
    have locked.
    The p/n sequence is of length 64 for both the {\tt~frame64} and
    {\tt~flexframe} structures (63-bit $m$-sequence with additional padded
    bit).
\item[{\sf header}]
    is a fixed-length data sequence which contains a small amount of
    information about the rest of the frame.
    The headers for the {\tt~frame64} and {\tt~flexframe} structures are
    vastly different and are described independently.
\item[{\sf payload}]
    is the meat of the frame, containing the raw data to be transferred across
    the link.
    For the {\tt~frame64} structure, the payload is fixed at 64 bytes (hence
    its moniker), encoded using the Hamming~(7,4) code
    (section~\ref{module:fec}), and modulated using QPSK.
    The {\tt~flexframe} structure has a variable length payload and can be
    modulated using whatever schemes the user desires, however forward
    error-correction is executed externally.
    In both cases the synchronizer object invokes the callback upon receiving
    the payload.
\item[{\sf ramp/down}]
    gracefully decreases the output signal level as per ramp/up.
\end{description}

NOTE: while the {\tt flexframegen} and {\tt flexframesync} objects are
intended to be used in conjunction with one another, the output of
{\tt flexframegen} requires matched-filtering interpolation before the
{\tt~flexframesync} object can recover the data.



% 
% MODULE : interleaver
%

\section{interleaver}
\label{module:interleaver}

This section describes the functionality of the \liquid\
{\tt interleaver} object.
Interleavers serve to distribute grouped bit errors evenly throughout a block
of data.
This aids certain forward error-correction (FEC) codes in correcting
bit errors (see section~\ref{module:fec} on error-correcting codes).
On the transmit side of the wireless link, the interleaver re-orders the bits
after FEC encoding and before modulation.
On the receiving side, the de-interleaver re-shuffles the bits to their
original position before attempting to run the FEC decoder.

%The {\tt interleaver} object

The bit-shuffling order must be known at both the transmitter and receiver.
Two options are available in \liquid\ for shuffling bits:
{\tt LIQUID\_INTERLEAVER\_BLOCK} and {\tt LIQUID\_INTERLEAVER\_SEQUENCE}.

\subsection{{\tt LIQUID\_INTERLEAVER\_BLOCK} (block interleaving)}
\label{module:interleaver:block}

\subsection{{\tt LIQUID\_INTERLEAVER\_SEQUENCE} ($m$-sequence interleaving)}
\label{module:interleaver:sequence}
See also {\tt msequence} (section~\ref{module:sequence}).

\subsection{interface}
\label{module:interleaver:interface}

%-------------------- FIGURE: interleaver scatterplot --------------------
\begin{figure}
\centering
\mbox{
  \subfigure[$i=0$]
    {
      \includegraphics[trim = 15mm 0mm 15mm 0mm, clip, height=6cm]{figures.gen/interleaver_scatterplot_i0}
      \label{fig:interleaver:scatterplot:0}
    } \quad
  \subfigure[$i=1$]
    {
      \includegraphics[trim = 15mm 0mm 15mm 0mm, clip, height=6cm]{figures.gen/interleaver_scatterplot_i1}
      \label{fig:interleaver:scatterplot:1}
    } \quad
}
\mbox{
  \subfigure[$i=2$]
    {
      \includegraphics[trim = 15mm 0mm 15mm 0mm, clip, height=6cm]{figures.gen/interleaver_scatterplot_i2}
      \label{fig:interleaver:scatterplot:2}
    } \quad
  \subfigure[$i=3$]
    {
      \includegraphics[trim = 15mm 0mm 15mm 0mm, clip, height=6cm]{figures.gen/interleaver_scatterplot_i3}
      \label{fig:interleaver:scatterplot:3}
    } \quad
}
\caption{{\tt interleaver} demonstration of a 64-byte (512-bit) sequence with
increasing number of iterations (interleaving depth)}
\label{fig:module:interleaver:scatterplot}
\end{figure}



% 
% MODULE : math
%

\section{math}
\label{module:math}
transcendental functions not in the C standard library (gamma, besseli, etc.)
and polynomial operations

{\tt liquid\_lngammaf}
\[
    \ln(\Gamma(z)) \approx
    \frac{z}{2} \ln\left( \frac{2\pi}{z} \right)
    \left(
        \ln\left(z + \frac{1}{12 z - 0.1/z} \right) - 1
    \right)
\]

{\tt liquid\_sincf}
\[ \sinc(z) = \frac{\sin(z)}{2\pi z} \]
For small $z$, this can be approximated as
\[
    \sinc(z) \approx \prod_{k=1}^{\infty}{ \cos\left( 2^{-k} \pi z \right) }
\]


% 
% MODULE : matrix
%
\section{matrix}
\label{module:matrix}
Matrices in {\it liquid} are represented as just arrays of a single dimension,
and do not rely on special objects for their manipulation.
\input{listings/matrix.example.c.tex}

\subsection{Basic math operations}
\label{module:matrix:math}
add, sub, mul, div, trans(pose), eye

\subsubsection{{\tt matrix\_eye} (identity matrix)}
The {\tt matrix\_eye} method generates the $n \times n$ identity matrix.
\[
    \vec{I}_n = 
    \begin{bmatrix}
        1 & 0 & \cdots & 0 \\
        0 & 1 & \cdots & 0 \\
        %  &   &        &   \\
        \\
        0 & 0 & \cdots & 1 \\
    \end{bmatrix}
\]

\subsection{Elementary math operations}
\label{module:matrix:elementary}
pivot, swaprows

Swap rows...
\begin{verbatim}
x = 
  0.84381998 -2.38303995  1.43060994 -1.66603994
  3.99475002  0.88066000  4.69372988  0.44563001
  7.28072023 -2.06608009  0.67074001  9.80657005
  6.07741022 -3.93098998  1.22826004 -0.42142001

matrixf_swaprows(x,4,4,0,2);
  7.28072023 -2.06608009  0.67074001  9.80657005
  3.99475002  0.88066000  4.69372988  0.44563001
  0.84381998 -2.38303995  1.43060994 -1.66603994
  6.07741022 -3.93098998  1.22826004 -0.42142001
\end{verbatim}

\subsubsection{Pivoting}
[NOTE: terminology for ``pivot'' is different from literature.]
Given an $n \times m$ matrix $\vec{A}$...
\[
    \vec{A} = 
    \begin{bmatrix}
        A_{0,0}     & A_{0,1}   & \cdots  & A_{0,m-1} \\
        A_{1,0}     & A_{1,1}   & \cdots  & A_{1,m-1} \\
        \\
        A_{n-1,0}   & A_{n-1,1} & \cdots  & A_{n-1,m-1}
    \end{bmatrix}
\]
The pivot element must not be zero.
Pivoting $\vec{A}$ around $\vec{A}_{a,b}$ gives
\[
    \vec{B}_{i,j} = \left(
                    \frac{\vec{A}_{i,b}}{\vec{A}_{a,b}}
                    \right)
                    \vec{A}_{a,j} - \vec{A}_{i,j}
                    \forall i \ne a
\]
Row $a$ is left unchanged in $\vec{B}$.
All elements of $\vec{B}$ in column $b$ are zero except for row $a$.
For our example $4 \times 4$ matrix $\vec{x}$, pivoting around
$\vec{x}_{1,2}$ gives:
\begin{verbatim}
matrixf_pivot(x,4,4,1,2);
  0.37374675  2.65145779  0.00000000  1.80186427
  3.99475002  0.88066000  4.69372988  0.44563001
 -6.70986557  2.19192743  0.00000000 -9.74288940
 -5.03205967  4.16144180  0.00000000  0.53803295
\end{verbatim}

\subsection{Complex math operations}
\label{module:matrix:complex}
inv(ert), gjelim (Gauss-Jordan elimination)

\subsubsection{Inverse}
Given an $n \times n$ matrix $\vec{A}$, augment with $\vec{I}_n$:
\[
    \left[\vec{A}|\vec{I}_n\right] = 
    \left[
    \begin{array}{cccc|cccc}
    A_{0,0}     & A_{0,1}   & \cdots  & A_{0,m-1}   & 1 & 0 & \cdots & 0 \\
    A_{1,0}     & A_{1,1}   & \cdots  & A_{1,m-1}   & 0 & 1 & \cdots & 0 \\
                &           &         &             &   &   &        &   \\
    A_{n-1,0}   & A_{n-1,1} & \cdots  & A_{n-1,m-1} & 0 & 0 & \cdots & 1 \\
    \end{array}
    \right]
\]
Perform elementary operations to convert to its row-reduced echelon form.
The resulting matrix has the identity matrix on the left and $\vec{A}^{-1}$ on
its right, viz
\[
    \left[\vec{I}_n|\vec{A}^{-1}\right] = 
    \left[
    \begin{array}{cccc|cccc}
1 & 0 & \cdots & 0 & A^{-1}_{0,0}   & A^{-1}_{0,1}   & \cdots  & A^{-1}_{0,m-1}   \\
0 & 1 & \cdots & 0 & A^{-1}_{1,0}   & A^{-1}_{1,1}   & \cdots  & A^{-1}_{1,m-1}   \\
  &   &        &   &                &                &         &                  \\
0 & 0 & \cdots & 1 & A^{-1}_{n-1,0} & A^{-1}_{n-1,1} & \cdots  & A^{-1}_{n-1,m-1} \\
    \end{array}
    \right]
\]
The {\tt matrix\_inv} method uses Gauss-Jordan elmination (see 
{\tt matrix\_gjelim}) for row reduction and back-substitution.
Pivot elements in $\vec{A}$ with the largest magnitude are chosen to help
stability in floating-point arithmetic.
\begin{verbatim}
matrixf_inv(x,4,4);
 -0.33453920  0.04643385 -0.04868321  0.23879384
 -0.42204019  0.12152659 -0.07431178  0.06774280
  0.35104612  0.15256262  0.04403552 -0.20177667
  0.13544561 -0.01930523  0.11944833 -0.14921521
\end{verbatim}

\subsubsection{Determinant}
The determinant of an $n \times n$ matrix $\vec{A}$...
In {\it liquid}, the determinant is computed by L/U decomposition of $\vec{A}$
using Doolittle's method (see {\tt matrix\_ludecomp\_doolittle}) and then
computing the product of the diagonal elements of $\vec{U}$, viz
\[
    \det\left(\vec{A}\right) =
    \left|\vec{A}\right| =
    \prod_{k=0}^{n-1}{\vec{U}_{k,k}}
\]
This is equivalent to performing L/U decomposition using Crout's method and
then computing the product of the diagonal elements of $\vec{L}$.
\begin{verbatim}
matrixf_det(X,4,4) = 585.40289307
\end{verbatim}

\subsubsection{LU Decomposition, Crout's Method}
Crout's method decomposes a non-singular $n\times n$ matrix $\vec{A}$ into a
product of a lower triangular $n \times n$ matrix $\vec{L}$ and an upper
triangular $n \times n$ matrix $\vec{U}$. %NOTE : discuss permutation matrix P
In fact, $\vec{U}$ is a unit upper triangular matrix (its values along the
diagonal are 1).

\[
    \vec{L}_{i,k} = \vec{A}_{i,k} -
                    \sum_{t=0}^{k-1}{ \vec{L}_{i,t} \vec{U}_{t,k} }
                    \ \forall k \in \{0,n-1\}, i \in \{k,n-1\}
\]

\[
    \vec{U}_{k,j} = \left[
                    \vec{A}_{k,j} -
                    \sum_{t=0}^{k-1}{ \vec{L}_{k,t} \vec{U}_{t,j} }
                    \right] / \vec{L}_{k,k}
                    \ \forall k \in \{0,n-1\}, j \in \{k+1,n-1\}
\]

\begin{verbatim}
matrixf_ludecomp_crout(X,4,4,L,U,P)
L =
  0.84381998  0.00000000  0.00000000  0.00000000
  3.99475002 12.16227055  0.00000000  0.00000000
  7.28072023 18.49547005 -8.51144791  0.00000000
  6.07741022 13.23228073 -6.81350422 -6.70173073
U =
  1.00000000 -2.82410932  1.69539714 -1.97440207
  0.00000000  1.00000000 -0.17093502  0.68514121
  0.00000000  0.00000000  1.00000000 -1.35225296
  0.00000000  0.00000000  0.00000000  1.00000000
\end{verbatim}

Doolittle's method...
\begin{verbatim}
matrixf_ludecomp_doolittle(X,4,4,L,U,P)
L =
  1.00000000  0.00000000  0.00000000  0.00000000
  4.73412609  1.00000000  0.00000000  0.00000000
  8.62828636  1.52072513  1.00000000  0.00000000
  7.20225906  1.08797777  0.80051047  1.00000000
U =
  0.84381998 -2.38303995  1.43060994 -1.66603994
  0.00000000 12.16227150 -2.07895803  8.33287334
  0.00000000  0.00000000 -8.51144791 11.50963116
  0.00000000  0.00000000  0.00000000 -6.70172977
\end{verbatim}

\subsubsection{Gauss-Jordan Elimination}
Gauss-Jordan elimination converts a $n \times m$ matrix into its row-reduced
echelon form using elementary matrix operations (e.g. pivoting).
This can be used to solve a linear system of $n$ equations
$\vec{A}\vec{x} = \vec{b}$ for the unknown vector $\vec{x}$
\[
    \begin{bmatrix}
        A_{0,0}     & A_{0,1}   & \cdots  & A_{0,n-1} \\
        A_{1,0}     & A_{1,1}   & \cdots  & A_{1,n-1} \\
        \\
        A_{n-1,0}   & A_{n-1,1} & \cdots  & A_{n-1,n-1}
    \end{bmatrix}
    \begin{bmatrix}
        x_{0} \\
        x_{1} \\
        \\
        x_{n-1}
    \end{bmatrix}
    =
    \begin{bmatrix}
        b_{0} \\
        b_{1} \\
        \\
        b_{n-1}
    \end{bmatrix}
\]
The solution for $\vec{x}$ is given by inverting $\vec{A}$ and multiplying
by $\vec{b}$, viz
\[
    \vec{x} = \vec{A}^{-1}\vec{b}
\]
This is also equivalent to augmenting $\vec{A}$ with $\vec{b}$ and
converting it to its row-reduced echelon form.
If $\vec{A}$ is non-singular the resulting $n \times n+1$ matrix will hold
$\vec{x}$ in its last column.
The row-reduced echelon form of a matrix is computed in {\it liquid} using the
Gauss-Jordan elimination algorithm, and can be invoked as such:
\begin{verbatim}
Ab =
  0.84381998 -2.38303995  1.43060994 -1.66603994  0.91488999
  3.99475002  0.88066000  4.69372988  0.44563001  0.71789002
  7.28072023 -2.06608009  0.67074001  9.80657005  1.06552994
  6.07741022 -3.93098998  1.22826004 -0.42142001 -0.81707001
matrixf_gjelim(Ab,4,5)
  1.00000000 -0.00000000  0.00000000 -0.00000000 -0.51971692
 -0.00000000  1.00000000  0.00000000  0.00000000 -0.43340963
 -0.00000000 -0.00000000  1.00000000 -0.00000000  0.64247853
  0.00000000 -0.00000000 -0.00000000  0.99999994  0.35925382
\end{verbatim}
Notice that the result contains $\vec{I}_n$ in its first $n$ rows and $n$
columns (to within machine precision).
[NOTE: row permutations (swapping) might have occurred...]


% 
% MODULE : modem
%

\section{modem}
\label{module:modem}
The modem module implements a set of (mod)ulation/(dem)odulation schemes
for encoding information into signals.
%For the analog modems, samples are modulated according to...
For the digital modems, data bits are encoded into symbols representing
carrier frequency, phase, amplitude, etc.

\subsection{Analog modulation schemes}
freqmod, ampmod, etc.

\subsection{Continuous phase digital modulation schemes}
fsk, msk, etc.

\subsection{Linear digital modulation schemes}
(d)psk, apsk, ask, qam

The {\tt modem} object realizes the linear digital modulation library in which
the information from a symbol is encoded into the amplitude and phase of a
sample.
The input/output relationship for modulation/demodulation for {\tt modem} is
strictly one-to-one and is independent of any pulse shaping, or interpolation.
This differs from {\tt cpmodem} in which the pulse shaping filter is
integrated into the modem itself.

\subsubsection{Usage}
\begin{itemize}
\item[] {\tt modem\_modulate()} converts an input symbol into an output sample
\item[] {\tt modem\_demodulate()} finds the closest symbol which matches the
input sample
\item[] {\tt modem\_get\_demodulator\_phase\_error()} returns an angle
proportional to the phase error after demodulation
\item[] {\tt modem\_get\_demodulator\_evm()} returns a value equal to the
error vector magnitude after demodulation
\end{itemize}

\subsubsection{{\tt MOD\_PSK}, {\tt MOD\_DPSK}}
Phase-shift keying...

%-------------------- FIGURE: PSK MODEM --------------------
\begin{figure}[ht]
\centering
\mbox{
  \subfigure[BPSK]
    {
      \includegraphics[trim = 15mm 0mm 15mm 0mm, clip, height=6cm]{figures.gen/modem_bpsk}
      \label{fig:modem:psk:2}
    } \quad
  \subfigure[QPSK]
    {
      \includegraphics[trim = 15mm 0mm 15mm 0mm, clip, height=6cm]{figures.gen/modem_qpsk}
      \label{fig:modem:psk:4}
    } \quad
}
\mbox{
  \subfigure[8-PSK]
    {
      \includegraphics[trim = 15mm 0mm 15mm 0mm, clip, height=6cm]{figures.gen/modem_8psk}
      \label{fig:modem:psk:8}
    } \quad
  \subfigure[16-PSK]
    {
      \includegraphics[trim = 15mm 0mm 15mm 0mm, clip, height=6cm]{figures.gen/modem_16psk}
      \label{fig:modem:psk:16}
    } \quad
}
% trim = left bottom right top
\caption{Phase-shift keying (PSK) modem}
\label{fig:modem:psk}
\end{figure}


%-------------------- FIGURE: APSK MODEM --------------------
\begin{figure}[ht]
\centering
\mbox{
  \subfigure[4-APSK (1,3)]
    {
      \includegraphics[trim = 15mm 0mm 15mm 0mm, clip, height=6cm]{figures.gen/modem_4apsk}
      \label{fig:modem:apsk:4}
    } \quad
  \subfigure[8-PSK (1,7)]
    {
      \includegraphics[trim = 15mm 0mm 15mm 0mm, clip, height=6cm]{figures.gen/modem_8apsk}
      \label{fig:modem:apsk:8}
    } \quad
}
\mbox{
  \subfigure[16-APSK (4,12)]
    {
      \includegraphics[trim = 15mm 0mm 15mm 0mm, clip, height=6cm]{figures.gen/modem_16apsk}
      \label{fig:modem:apsk:16}
    } \quad
  \subfigure[32-APSK (4,12,16)]
    {
      \includegraphics[trim = 15mm 0mm 15mm 0mm, clip, height=6cm]{figures.gen/modem_32apsk}
      \label{fig:modem:apsk:32}
    } \quad
}
\mbox{
  \subfigure[64-APSK (4,14,20,26)]
    {
      \includegraphics[trim = 15mm 0mm 15mm 0mm, clip, height=6cm]{figures.gen/modem_64apsk}
      \label{fig:modem:apsk:64}
    } \quad
  \subfigure[128-APSK (8,18,24,36,42)]
    {
      \includegraphics[trim = 15mm 0mm 15mm 0mm, clip, height=6cm]{figures.gen/modem_128apsk}
      \label{fig:modem:apsk:128}
    } \quad
}
% trim = left bottom right top
\caption{Amplitude/phase-shift keying (APSK) modem}
\label{fig:modem:apsk}
\end{figure}


%-------------------- FIGURE: ASK MODEM --------------------
\begin{figure}[ht]
\centering
\mbox{
  \subfigure[2-ASK]
    {
      \includegraphics[trim = 15mm 0mm 15mm 0mm, clip, height=6cm]{figures.gen/modem_2ask}
      \label{fig:modem:ask:2}
    } \quad
  \subfigure[4-ASK]
    {
      \includegraphics[trim = 15mm 0mm 15mm 0mm, clip, height=6cm]{figures.gen/modem_4ask}
      \label{fig:modem:ask:4}
    } \quad
}
\mbox{
  \subfigure[8-ASK]
    {
      \includegraphics[trim = 15mm 0mm 15mm 0mm, clip, height=6cm]{figures.gen/modem_8ask}
      \label{fig:modem:ask:8}
    } \quad
  \subfigure[16-ASK]
    {
      \includegraphics[trim = 15mm 0mm 15mm 0mm, clip, height=6cm]{figures.gen/modem_16ask}
      \label{fig:modem:ask:16}
    } \quad
}
% trim = left bottom right top
\caption{Pulse-amplitude modulation (ASK) modem}
\label{fig:modem:ask}
\end{figure}


%-------------------- FIGURE: QAM MODEM --------------------
\begin{figure}[ht]
\centering
\mbox{
  \subfigure[8-QAM]
    {
      \includegraphics[trim = 15mm 0mm 15mm 0mm, clip, height=6cm]{figures.gen/modem_8qam}
      \label{fig:modem:qam:8}
    } \quad
  \subfigure[16-QAM]
    {
      \includegraphics[trim = 15mm 0mm 15mm 0mm, clip, height=6cm]{figures.gen/modem_16qam}
      \label{fig:modem:qam:16}
    } \quad
}
\mbox{
  \subfigure[32-QAM]
    {
      \includegraphics[trim = 15mm 0mm 15mm 0mm, clip, height=6cm]{figures.gen/modem_32qam}
      \label{fig:modem:qam:32}
    } \quad
  \subfigure[64-QAM]
    {
      \includegraphics[trim = 15mm 0mm 15mm 0mm, clip, height=6cm]{figures.gen/modem_64qam}
      \label{fig:modem:qam:64}
    } \quad
}
\mbox{
  \subfigure[128-QAM]
    {
      \includegraphics[trim = 15mm 0mm 15mm 0mm, clip, height=6cm]{figures.gen/modem_128qam}
      \label{fig:modem:qam:128}
    } \quad
  \subfigure[256-QAM]
    {
      \includegraphics[trim = 15mm 0mm 15mm 0mm, clip, height=6cm]{figures.gen/modem_256qam}
      \label{fig:modem:qam:256}
    } \quad
}
% trim = left bottom right top
\caption{Rectangular quaternary-amplitude modulation (QAM) modem}
\label{fig:modem:qam}
\end{figure}



%
% MODULE : multicarrier
%

\newpage
\section{multicarrier}
\label{module:multicarrier}

\subsection{{\tt firpfbch}}
\label{module:multicarrier:firpfbch}
finite impulse response polyphase filterbank channelizer (firpfbch)
\begin{itemize}
\item uses FFTW library (www.fftw.org) if available, internal FFT library
      otherwise
\item basis behind OFDM/OQAM
\end{itemize}


\subsection{{\tt ofdmframe}}
\label{module:multicarrier:ofdmframe}
orthogonal frequency-divisional multiplexing (OFDM) framing structure.
\begin{itemize}
\item uses FFTW library (www.fftw.org) if available, internal FFT library
      otherwise
\item useful for OFDM communications
\item allows spectrum notching
\item not anticipated to be used with specific standards
\end{itemize}




% 
% MODULE : nco (numerically-controlled oscillator)
%

\section{nco (numerically-controlled oscillator)}
\label{module:nco}
numerically-controlled oscillator: mix, pll

\subsection{{nco}}
\label{module:nco:nco}
The {\tt nco} object implements an oscillator with two options for internal
phase precision: {\tt LIQUID\_NCO} and {\tt LIQUID\_VCO}.

\subsection{{pll} (phase-locked loop)}
\label{module:nco:pll}
The phase-locked loop object provides a method for synchronizing oscillators
on different platforms.
It uses a second-order integrating loop filter to adjust the frequency of its
{\tt nco} based on an instantaneous phase error input.

For a given bandwidth $b$, damping factor $\xi$, and scaling factor $k_1$,
the loop filter coefficients are
\[  \beta = \frac{2b}{k_1} \left(\xi + \frac{1}{4\xi} \right)   \]
\[  \alpha = 2 \xi \beta    \]
Given a phase error $v$, the filtered phase error is given by
\[  q^\prime(n) = v \beta q^\prime(n-1) \]
\[  v = \alpha v + q^\prime(n)          \]


% optimization documentation
\section{optim (optimization)}
Newton-raphson, gradient, evolutionary algorithms, etc.

\subsection{Gradient search}
This module implements a steepest-descent search or gradient search.
Given a function $f$ which operates on a vector
$\vec{x} = \{x_0,x_1,\ldots,x_{N-1}\}$ of $N$ parameters,
the gradient search method seeks to find the optimum $\vec{x}$ which
minimizes $f$.
The gradient search is iterative, and adjusts $\vec{x}$ proportional to the
negative of the gradient of $f$ evaluated at the current location.
\[
    \Delta \vec{x}[n+1] = -\gamma[n] \nabla f(\vec{x}[n])
\]
where $\gamma[n]$ is the step size and
$\nabla f(\vec{x}[n])$ is the gradient of $f$ at $\vec{x}$, at the $n^{th}$
iteration.
The gradient is a vector field which points to the greatest rate of increase,
and is computed as
\[
    \nabla f(\vec{x}) = \left(
        \frac{\partial f}{\partial x_0},
        \frac{\partial f}{\partial x_1},
        \ldots,
        \frac{\partial f}{\partial x_{N-1}}
    \right)
\]
In most non-linear optimization problems, $\nabla f(\vec{x})$ is not known,
and must be estimated for each value of $\vec{x}[n]$ using the finite element
method.
The $k^{th}$ component of the gradient is approximated by
\[
    \frac{\partial f(\vec{x})}{\partial x_k} \approx 
    \frac{f(x_0,\ldots,x_k+\Delta,\ldots,x_{N-1}) - f(\vec{x})}{\Delta}
\]
An additional filtering operation is introduced on the step size to help
convergence, and therefore the updated vector at time $n+1$ is
\[
    \vec{x}[n+1] = \alpha \Delta\vec{x}[n+1] + (1-\alpha)\Delta\vec{x}[n]
\]
where $\Delta\vec{x}[0] = \vec{0}$.
In {\it liquid}, the gradient is normalized to unity (orthonormal).
Furthermore, $\gamma$ is slightly reduced each epoch.

% gradient_search example
\input{listings/gradient_search.example.c.tex}



% 
% MODULE : quantization
%

\newpage
\section{quantization}
\label{module:quantization}
analog/digital converters, companding...

\subsection{Compression \& Expansion}
\label{module:quantization:companding}

\begin{figure}
\centering
  \includegraphics[trim = 16mm 0mm 18mm 0mm, clip, width=8cm]{figures.gen/quantization_compander}
\caption{Compander transfer function}
\label{fig:module:quantization:compander}
\end{figure}
%


\subsection{{\tt quantizer}}
\label{module:quantization:quantizer}
%
\begin{figure}
\centering
\mbox{
  \subfigure[$b=4$] {
      \includegraphics[trim = 16mm 0mm 18mm 0mm, clip, width=8cm]{figures.gen/quantization_adc_b4}
      \label{fig:module:quantization:adc_b4}
    } \quad
  \subfigure[$b=5$] {
      \includegraphics[trim = 16mm 0mm 18mm 0mm, clip, width=8cm]{figures.gen/quantization_adc_b5}
      \label{fig:module:quantization:adc_b5}
    } \quad
}
\caption{Quantization ADC transfer function.}
\label{fig:module:quantization:adc}
\end{figure}
%



% 
% MODULE : random
%

\section{random}
\label{module:random}
random number generators

\subsection{Uniform}
\[
    f_x(x) =
    \begin{cases}
        1 & \text{if $0 < x \le 1$} \\
        0 & \text{else}.
    \end{cases}
\]

\subsection{Normal}
\[
    f_x(x;\sigma,\eta) =
        \frac{1}{\sigma \sqrt{2 \pi}}
        e^{-\left(x-\eta\right)^2/{2\sigma^2}}
\]
Computed using the Box-Muller method...

\subsection{Weibull}
\[
    f_x(x;\alpha,\beta,\gamma) =
    \begin{cases}
        \alpha(x-\gamma)^(\beta-1)
        e^{-(\alpha/\beta)(x-\gamma)^\beta} & \text{$x \ge \gamma$} \\
        0 &                                   \text{else}.
    \end{cases}
\]
Generated by inverting the Weibull cumulative ditribution function.

%\subsection{Gamma}

\subsection{Rice-$K$}
\[
    f_x(x;K,\Omega) = \cdots
\]
Generated from two independent normal random variables.


% 
% MODULE : sequence
%

\section{sequence}
\label{module:sequence}
linear feedback shift registers, complementary codes, etc.


% 
% MODULE : utility
%

\newpage
\section{utility}
\label{module:utility}
The utility module contains useful functions, primarily for bit fast bit
manipulation.
This includes packing/unpacking byte arrays, counting ones in an integer,
computing a binary dot-product, and others.

\subsection{{\tt liquid\_pack\_bytes()},
            {\tt liquid\_unpack\_bytes()}, and
            {\tt liquid\_repack\_bytes()}}
\label{module:utility:pack_bytes}
Byte packing is used extensively in the
{\tt fec} (\S\ref{module:fec}) and
{\tt framing} (\S\ref{module:framing}) modules.
These methods resize symbols represented by various numbers of bits.
This is necessary to move between raw data arrays which use full bytes (eight
bits per symbol) to methods expecting symbols of different sizes.
In particular, the {\tt liquid\_repack\_bytes()} method is useful when one wants
to transmit a block of 64 data bytes using an 8-PSK modem which requires a
3-bit input symbol.
For example repacking two 8-bit symbols {\tt 00000000},{\tt 11111111} into six
3-bit symbols gives
{\tt 000},{\tt 000},{\tt 001},{\tt 111},{\tt 111},{\tt 100}.
Because 16 bits cannot be divided evenly among 3-bit symbols, the last symbol
is padded with zeros.

\subsection{{\tt liquid\_pack\_array()},
            {\tt liquid\_unpack\_array()}}
\label{module:utility:pack_array}
%
The {\tt liquid\_pack\_array()} and {\tt liquid\_unpack\_array()}
methods pack an array with symbols of arbitrary length.
These methods are similar to those in
\S\ref{module:utility:pack_bytes}
but are capable of packing symbols of any arbitrary length.
These are convenient for digital modulation and demodulation of a block
of symbols with different modulation schemes.
For example packing an array with five symbols
{\tt 1000},{\tt 011},{\tt 11010},{\tt 1},{\tt 000} yields two bytes:
{\tt 10000111,10101000}.
%
Here are the basic interfaces for packing and unpacking arrays:
\begin{Verbatim}[fontsize=\small]
  // pack binary array with symbol(s)
  void liquid_pack_array(unsigned char * _src,        // source array [size: _n x 1]
                         unsigned int _n,             // input source array length
                         unsigned int _k,             // bit index to write in _src
                         unsigned int _b,             // number of bits in input symbol
                         unsigned char _sym_in);      // input symbol

  // unpack symbols from binary array
  void liquid_unpack_array(unsigned char * _src,      // source array [size: _n x 1]
                           unsigned int _n,           // input source array length
                           unsigned int _k,           // bit index to write in _src
                           unsigned int _b,           // number of bits in output symbol
                           unsigned char * _sym_out); // output symbol
\end{Verbatim}
%
Listed below is a simple example of packing symbols of varying lengths
into a fixed array of bytes;
%
\input{listings/pack_array.example.c.tex}
%


\subsection{{\tt liquid\_lbshift()},
            {\tt liquid\_rbshift()}}
\label{module:utility:bshift}
Binary shifting.

{\tt liquid\_lbshift()}
\begin{Verbatim}[fontsize=\small]
    // input        : 1000 0001 1110 1111 0101 1111 1010 1010
    // output [0]   : 1000 0001 1110 1111 0101 1111 1010 1010
    // output [1]   : 0000 0011 1101 1110 1011 1111 0101 0100
    // output [2]   : 0000 0111 1011 1101 0111 1110 1010 1000
    // output [3]   : 0000 1111 0111 1010 1111 1101 0101 0000
    // output [4]   : 0001 1110 1111 0101 1111 1010 1010 0000
    // output [5]   : 0011 1101 1110 1011 1111 0101 0100 0000
    // output [6]   : 0111 1011 1101 0111 1110 1010 1000 0000
    // output [7]   : 1111 0111 1010 1111 1101 0101 0000 0000
\end{Verbatim}

{\tt liquid\_rbshift()}
\begin{Verbatim}[fontsize=\small]
    // input        : 1000 0001 1110 1111 0101 1111 1010 1010
    // output [0]   : 1000 0001 1110 1111 0101 1111 1010 1010
    // output [1]   : 0100 0000 1111 0111 1010 1111 1101 0101
    // output [2]   : 0010 0000 0111 1011 1101 0111 1110 1010
    // output [3]   : 0001 0000 0011 1101 1110 1011 1111 0101
    // output [4]   : 0000 1000 0001 1110 1111 0101 1111 1010
    // output [5]   : 0000 0100 0000 1111 0111 1010 1111 1101
    // output [6]   : 0000 0010 0000 0111 1011 1101 0111 1110
    // output [7]   : 0000 0001 0000 0011 1101 1110 1011 1111
\end{Verbatim}


\subsection{{\tt liquid\_lbcircshift()},
            {\tt liquid\_rbcircshift()}}
\label{module:utility:bcircshift}
Binary circular shifting.

{\tt liquid\_lbcircshift()}
\begin{Verbatim}[fontsize=\small]
    // input        : 1001 0001 1110 1111 0101 1111 1010 1010
    // output [0]   : 1001 0001 1110 1111 0101 1111 1010 1010
    // output [1]   : 0010 0011 1101 1110 1011 1111 0101 0101
    // output [2]   : 0100 0111 1011 1101 0111 1110 1010 1010
    // output [3]   : 1000 1111 0111 1010 1111 1101 0101 0100
    // output [4]   : 0001 1110 1111 0101 1111 1010 1010 1001
    // output [5]   : 0011 1101 1110 1011 1111 0101 0101 0010
    // output [6]   : 0111 1011 1101 0111 1110 1010 1010 0100
    // output [7]   : 1111 0111 1010 1111 1101 0101 0100 1000
\end{Verbatim}


{\tt liquid\_rbcircshift()}
\begin{Verbatim}[fontsize=\small]
    // input        : 1001 0001 1110 1111 0101 1111 1010 1010
    // output [0]   : 1001 0001 1110 1111 0101 1111 1010 1010
    // output [1]   : 0100 1000 1111 0111 1010 1111 1101 0101
    // output [2]   : 1010 0100 0111 1011 1101 0111 1110 1010
    // output [3]   : 0101 0010 0011 1101 1110 1011 1111 0101
    // output [4]   : 1010 1001 0001 1110 1111 0101 1111 1010
    // output [5]   : 0101 0100 1000 1111 0111 1010 1111 1101
    // output [6]   : 1010 1010 0100 0111 1011 1101 0111 1110
    // output [7]   : 0101 0101 0010 0011 1101 1110 1011 1111
\end{Verbatim}

\subsection{{\tt liquid\_lshift()},
            {\tt liquid\_rshift()}}
\label{module:utility:shift}
Byte-wise shifting.

{\tt liquid\_lshift()}
\begin{Verbatim}[fontsize=\small]
    // input        : 1000 0001 1110 1111 0101 1111 1010 1010
    // output [0]   : 1000 0001 1110 1111 0101 1111 1010 1010
    // output [1]   : 1110 1111 0101 1111 1010 1010 0000 0000
    // output [2]   : 0101 1111 1010 1010 0000 0000 0000 0000
    // output [3]   : 1010 1010 0000 0000 0000 0000 0000 0000
    // output [4]   : 0000 0000 0000 0000 0000 0000 0000 0000
\end{Verbatim}


{\tt liquid\_rshift()}
\begin{Verbatim}[fontsize=\small]
    // input        : 1000 0001 1110 1111 0101 1111 1010 1010
    // output [0]   : 1000 0001 1110 1111 0101 1111 1010 1010
    // output [1]   : 0000 0000 1000 0001 1110 1111 0101 1111
    // output [2]   : 0000 0000 0000 0000 1000 0001 1110 1111
    // output [3]   : 0000 0000 0000 0000 0000 0000 1000 0001
    // output [4]   : 0000 0000 0000 0000 0000 0000 0000 0000
\end{Verbatim}

\subsection{{\tt liquid\_lcircshift()},
            {\tt liquid\_rcircshift()}}
\label{module:utility:circshift}
Byte-wise circular shifting.

{\tt liquid\_lcircshift()}
\begin{Verbatim}[fontsize=\small]
    // input        : 1000 0001 1110 1111 0101 1111 1010 1010
    // output [0]   : 1000 0001 1110 1111 0101 1111 1010 1010
    // output [1]   : 1110 1111 0101 1111 1010 1010 1000 0001
    // output [2]   : 0101 1111 1010 1010 1000 0001 1110 1111
    // output [3]   : 1010 1010 1000 0001 1110 1111 0101 1111
    // output [4]   : 1000 0001 1110 1111 0101 1111 1010 1010
\end{Verbatim}

{\tt liquid\_rcircshift()}
\begin{Verbatim}[fontsize=\small]
    // input        : 1000 0001 1110 1111 0101 1111 1010 1010
    // output [0]   : 1000 0001 1110 1111 0101 1111 1010 1010
    // output [1]   : 1010 1010 1000 0001 1110 1111 0101 1111
    // output [2]   : 0101 1111 1010 1010 1000 0001 1110 1111
    // output [3]   : 1110 1111 0101 1111 1010 1010 1000 0001
    // output [4]   : 1000 0001 1110 1111 0101 1111 1010 1010
\end{Verbatim}

\subsection{miscellany}
\label{module:utility:misc}
This section describes the bit-counting methods which are used extensively
throughout \liquid, particularly the
{\tt fec} (\S\ref{module:fec}) and
{\tt sequence} (\S\ref{module:sequence}) modules.
Integer sizes vary for different machines;
when \liquid\ is initially configured (see Chapter~\ref{section:installation}), the
size of the integer is computed such that the fastest method can be computed
without performing unnecessary loop iterations or comparisons.

\begin{description}
\item[{\tt liquid\_count\_ones(x)}]
    counts the number of {\tt 1}s that exist in the integer $x$.
    For example, the number {\tt 237} is represented in binary as
    {\tt 11101101}, therefore {\tt liquid\_count\_ones(237)} returns {\tt 6}.
\item[{\tt liquid\_count\_ones\_mod2(x)}]
    counts the number of {\tt 1}s that exist in the integer $x$, modulo 2; in
    other words, it returns {\tt 1} if the number of ones in $x$ is odd,
    {\tt 0} if the number is even.
    For example, {\tt liquid\_count\_ones\_mod2(237)} return {\tt 0}.
\item[{\tt liquid\_bdotprod(x,y)}]
    computes the binary dot-product between two integers $x$ and $y$ as the sum
    of ones in $x \land y$, modulo 2 (where $\land$ is the logical `and'
    operation).
    This is useful in linear feedback shift registers
    (see \S\ref{module:sequence:msequence} on $m$-sequences)
    as well as certain forward error-correction codes
    (see \S\ref{module:fec:hamming} on Hamming codes).
    For example, the binary dot product between
    {\tt 10110011} and
    {\tt 11101110} is
    {\tt 1} because
    {\tt 10110011} $\land$ {\tt 11101110} $=$ {\tt 10100010} which has an odd
    number of {\tt 1}s.
\item[{\tt liquid\_count\_leading\_zeros(x)}]
    counts the number of leading zeros in the integer $x$.
    This is dependent upon the size of the integer for the target machine
    which is usually either two or four bytes.
\item[{\tt liquid\_msb\_index(x)}]
    computes the index of the most-significant bit in the integer $x$.
    The function will return {\tt 0} for $x=0$.
    For example if $x=129$ ({\tt 10000001}), the function will return {\tt 8}.
\end{description}





%%%%%%%%%%%%%%%%%%%%%%%%%%%%%%%%%%%%%%%%%%%%%%%%%%%%%%%%%%%%%%%%%%%%%
%
%             Installation
%
%%%%%%%%%%%%%%%%%%%%%%%%%%%%%%%%%%%%%%%%%%%%%%%%%%%%%%%%%%%%%%%%%%%%%

\newpage
\part{Installation}
\label{part:installation}

%\bigskip
%\noindent
%blah blah blah...

% TODO throw up some fancy graphic here

% 
% installation
%
 
The basic installation is as follows:
\begin{verbatim}
    $ ./reconf
    $ ./configure
    $ make
    # make install
\end{verbatim}
The {\tt install} target copies the global header file {\tt liquid.h} and the
shared object file {\tt libliquid.a} to the system directories.
You must have root access to run make with the {\tt install} target.
Additionally, the {\tt uninstall} target removes these files from the system
directory.

The additional makefile targets are listed here.

\section{Building \& Dependencies}
\label{ch:installation:building}
Requirements:
\begin{itemize}
\item {\tt gcc}, the GNU compiler collection
\item {\tt libc}, the standard C library
\item {\tt libm}, the standard math library (eventually will be phased out to
optional)
\end{itemize}

\liquid\ was designed to be portable, and in doing so requires minimal dependencies to
build and run.
%Several well-known DSP packages (such as {\tt fftw} [cite] and {\it libfec} [cite])...
The project, however, will take advantage of other libraries if they are installed on the
target machine.
These optional packages are:
\begin{itemize}
\item {\tt fftw3}
\item {\tt libfec}
\item {\tt liquid-fpm} (liquid fixed-point math library)
\end{itemize}
The build system checks to see if they are installed during the {\tt configure} process
and will generate an appropriate {\tt config.h} if they are.

\subsection{Source code organization}
In order to keep the project relatively organized, the source code is broken
up into separate ``modules'' under the top {\tt src/} directory.

\section{Targets}
\label{ch:installation:targets}

\subsubsection{Modules}
Each module consists of a top-level included makefile, a {\tt README}, the
library source files, a set of test scripts, and a set of benchmarks.

\subsubsection{Examples ({\tt make examples})}
All examples are built as stand-alone programs not build by the target
{\tt all} by default.
All examples can be built by invoking {\tt make examples}, or a specific
example can be build by invoking e.g. {\tt make examples/modem\_example}.

\subsection{Autotests ({\tt make check})}
\label{ch:installation:targets:autotests}
Source code validation is a critical step in any software library,
particularly for verifying the portability of code to different processors and
platforms.
Packaged with \liquid\ are a number of automatic test scripts to validate the
correctness of the source code.
The test scripts are located under each module's {\tt tests} directory and
take the form of a C header file.
The testing framework operates similarly to CppUnit \cite{cppunit:web} and
cxxtest \cite{cxxtest:web}, however it is written in C (except for the
parser which, at the time of writing this, is in Python).
The python script {\tt autotest\_gen.py} parses these header files looking for
the key ``{\tt void autotest\_}'' which corresponds to a specific test.
%Each test that is found is appended to a list
The python script generates the header file {\tt autotest\_include.h} which
includes all the modules' test headers as well as several organizing
structures for keeping track of which tests have passed or failed.
The result is an executable file, {\tt xautotest}, which can be run to
validate the functional correctess of \liquid\ on your target platform.

\subsubsection{Macros}
Each module contains a number of autotest scripts which use pre-processor
macros for asserting the functional correctness of the souce code.

\begin{description}
\item[{\tt CONTEND\_EQUALITY}$(x,y)$] asserts that $x==y$ and fails if
false.
\item[{\tt CONTEND\_INEQUALITY}$(x,y)$] asserts that $x$ differs from
$y$.
\item[{\tt CONTEND\_GREATER\_THAN}$(x,y)$] asserts that $x>y$.
\item[{\tt CONTEND\_LESS\_THAN}$(x,y)$] asserts that $x<y$.
\item[{\tt CONTEND\_DELTA}$(x,y,\Delta)$] asserts thaat $|x-y|<\Delta$
\item[{\tt CONTEND\_EXPRESSION}$(expr)$] asserts that some expression is
true.
\item[{\tt CONTEND\_SAME\_DATA}$(ptrA,ptrB,n)$] asserts that each of $n$
byte values in the arrays referenced by $ptrA$ and $ptrB$ are equal.
\item[{\tt AUTOTEST\_PASS}$()$] passes unconditionally.
\item[{\tt AUTOTEST\_FAIL}$(string)$] prints $string$ and fails
unconditionaly.
\item[{\tt AUTOTEST\_WARN}$(string)$] simply prints a warning.
The autotest program will keep track of which tests elicit warnings and add
them to the list of unstable tests.
\end{description}

Here are some examples:
\begin{itemize}
\item[] {\tt CONTEND\_EQUALITY}(1,1) will {\it pass}
\item[] {\tt CONTEND\_EQUALITY}(1,2) will {\it fail}
\end{itemize}

\subsubsection{Running the autotests}
The result is an executable file named {\tt xautotest} which has several
options for running.
These options may be viewed with either the {\tt -h} or {\tt -u} flags (for
help/usage information).
\begin{verbatim}
    $ ./xautotest -h
    autotest options:
      -h,-u : prints this help file
      -t<n> : run specific test
      -p<n> : run specific package
      -L    : lists all autotests
      -l    : lists all packages
      -s    : stop on fail
      -v    : verbose
      -q    : quiet
\end{verbatim}
Simply running the program without any arguments executes all the tests and
displays the results to the screen.
The is the default response of the target {\tt make check}.

\subsection{Benchmarks ({\tt make bench})}
\label{ch:installation:targets:benchmarks}

\subsection{Targets summary}





%%%%%%%%%%%%%%%%%%%%%%%%%%%%%%%%%%%%%%%%%%%%%%%%%%%%%%%%%%%%%%%%%%%%%
%
%             BIBLIOGRAPHY
%
%%%%%%%%%%%%%%%%%%%%%%%%%%%%%%%%%%%%%%%%%%%%%%%%%%%%%%%%%%%%%%%%%%%%%
%\cleardoublepage
\bibliography{liquid}


% In LaTeX, each appendix is a "chapter"
\appendix
%%%%%%%%%%%%%%%%%%%%%%%%%%%%%%%%%%%%%%%%%%%%%%%%%%%%%%%%%%%%%%%%%%%%%
%
%             APPENDIX A
%
%%%%%%%%%%%%%%%%%%%%%%%%%%%%%%%%%%%%%%%%%%%%%%%%%%%%%%%%%%%%%%%%%%%%%
\end{document}

